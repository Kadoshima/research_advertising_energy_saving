% chapters/appx_b_assets.tex --- 付録B 実験資産一覧(ファイル・ディレクトリ)
\section{実験資産一覧}
\label{sec:assets}

\subsection{図の管理}
図は原則として\texttt{\detokenize{修士論文/figures/}}に集約し,本文から拡張子なしで参照する.本リポジトリでは,既存の解析結果図を相対パスで参照している箇所もあるため,提出用に固める際は\texttt{\detokenize{figures/}}へコピーしてパスを固定する.

\subsection{主要な出力先(例)}
\begin{itemize}
  \item 固定間隔の電力テーブル:\texttt{\detokenize{results/mhealth_policy_eval/power_table_sleep_eval_2025-12-14_interval_sweep_sleep_on_n9_10.csv}}
  \item オフライン評価(δ帯):\texttt{\detokenize{results/mhealth_policy_eval/letter_v4_scan90_v5_delta_tight_sleep_on_n9_10_actions_100_500/}}
  \item ストレス固定(v5図):\texttt{\detokenize{results/stress_fixed/figures_v5/}}
\end{itemize}

\subsection{実機評価(例:D2)}
実機の動的切替の例として,D2(scan90)では次のディレクトリ構成でログと指標を管理する.
\begin{itemize}
  \item 実験ルート:\texttt{\detokenize{uccs_d2_scan90/}}
  \item 入力(RX):\texttt{\detokenize{uccs_d2_scan90/data/RX/}}
  \item 入力(TXSD):\texttt{\detokenize{uccs_d2_scan90/data/TX/}}
  \item 出力(指標):\texttt{\detokenize{uccs_d2_scan90/metrics/01/summary.md}}
\end{itemize}

\subsection{解析スクリプト(例)}
解析は,1回で完結する巨大スクリプトではなく,「trial集計」「集約」「図表化」に分割して再現可能にする.代表例を以下に示す.
\begin{itemize}
  \item ストレス固定(trial集計):\texttt{\detokenize{scripts/analyze_stress_causal_real.py}}
  \item ストレス固定(集約):\texttt{\detokenize{scripts/aggregate_stress_causal_real_summary.py}}
  \item ストレス固定(v5図生成):\texttt{\detokenize{scripts/plot_stress_fixed_figures_v5.py}}
  \item D2集計:\texttt{\detokenize{uccs_d2_scan90/analysis/summarize_d2_run.py}}
\end{itemize}
