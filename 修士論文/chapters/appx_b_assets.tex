% chapters/appx_b_assets.tex --- 付録B 実験資産一覧(ファイル・ディレクトリ)
\section{実験資産一覧}
\label{sec:assets}

\subsection{図の管理}
図は原則として\texttt{\detokenize{修士論文/figures/}}に集約し,本文から拡張子なしで参照する.本リポジトリでは,既存の解析結果図を相対パスで参照している箇所もあるため,提出用に固める際は\texttt{\detokenize{figures/}}へコピーしてパスを固定する.

\subsection{主要な出力先(例)}
\begin{itemize}
  \item 固定間隔の電力テーブル:\texttt{\detokenize{results/mhealth_policy_eval/power_table_sleep_eval_2025-12-14_interval_sweep_sleep_on_n9_10.csv}}
  \item オフライン評価(制約帯):\texttt{\detokenize{results/mhealth_policy_eval/letter_v4_scan90_v5_delta_tight_sleep_on_n9_10_actions_100_500/}}
  \item ストレス固定(v5図):\texttt{\detokenize{results/stress_fixed/figures_v5/}}
  \item 追加検証(ブートストラップCI):\path{scripts/bootstrap_effects.py}
\end{itemize}

\subsection{実機評価(例:D2)}
実機の動的切替の例として,D2(scan90)では次のディレクトリ構成でログと指標を管理する.
\begin{itemize}
  \item 実験ルート:\path{uccs_d2_scan90/}
  \item 入力(run B/01, RX):\path{uccs_d2_scan90/data/B/01/RX/}
  \item 入力(run B/01, TXSD):\path{uccs_d2_scan90/data/B/01/TX/}
  \item 入力(追加 run B/02, RX):\path{uccs_d2_scan90/data/B/02/RX/}
  \item 入力(追加 run B/02, TXSD):\path{uccs_d2_scan90/data/B/02/TX/}
  \item 出力(指標,主結果・統合n=6):\path{uccs_d2_scan90/metrics/B_n6/summary.md}
  \item 出力(集計CSV):\path{uccs_d2_scan90/metrics/B_n6/summary_by_condition.csv}
  \item 出力(主張図):\path{uccs_d2_scan90/plots/d2b_B_n6_power_vs_pout.pdf} / \path{uccs_d2_scan90/plots/d2b_B_n6_power_vs_pout.png}
  \item 参考(run別n=3):\path{uccs_d2_scan90/metrics/B/summary.md} / \path{uccs_d2_scan90/metrics/B_02/summary.md}
\end{itemize}

\subsection{実機評価(例:D4)}
U-shuffleアブレーション(D4, scan90)では,次のディレクトリ構成でログと指標を管理する.
\begin{itemize}
  \item 実験ルート:\path{uccs_d4_scan90/}
  \item 入力(01, RX):\path{uccs_d4_scan90/data/01/RX/}
  \item 入力(01, TXSD):\path{uccs_d4_scan90/data/01/TX/}
  \item 出力(指標, n=3×4条件):\path{uccs_d4_scan90/metrics/01/summary.md}
  \item 出力(集計CSV):\path{uccs_d4_scan90/metrics/01/summary_by_condition.csv}
  \item 出力(主張図):\path{uccs_d4_scan90/plots/d4_01_power_vs_pout.pdf} / \path{uccs_d4_scan90/plots/d4_01_power_vs_pout.png}
\end{itemize}

\subsection{実機評価(例:D4B)}
CCS-offアブレーション(D4B, scan90)では,次のディレクトリ構成でログと指標を管理する.
\begin{itemize}
  \item 実験ルート:\path{uccs_d4b_scan90/}
  \item 入力(01, RX):\path{uccs_d4b_scan90/data/01/RX/}
  \item 入力(01, TXSD):\path{uccs_d4b_scan90/data/01/TX/}
  \item 出力(指標, n=3×4条件):\path{uccs_d4b_scan90/metrics/01/summary.md}
  \item 出力(集計CSV):\path{uccs_d4b_scan90/metrics/01/summary_by_condition.csv}
  \item 出力(効果量CI):\path{uccs_d4b_scan90/metrics/01/effects_ci.md} / \path{uccs_d4b_scan90/metrics/01/effects_ci.csv}
  \item 出力(主張図):\path{uccs_d4b_scan90/plots/d4b_01_power_vs_pout.pdf} / \path{uccs_d4b_scan90/plots/d4b_01_power_vs_pout.svg}
  \item 出力(統合図):\path{uccs_d4b_scan90/plots/role_separation_d3_d4_d4b.pdf} / \path{uccs_d4b_scan90/plots/role_separation_d3_d4_d4b.svg}
  \item 出力($\hat{\rho}_{100}$--$P_{\mathrm{out}}$俯瞰図):\path{uccs_d4b_scan90/plots/alpha_vs_pout_overview.pdf} / \path{uccs_d4b_scan90/plots/alpha_vs_pout_overview.svg}
  \item 出力(outage事例の追跡):\path{uccs_d4b_scan90/plots/outage_story_01/fig_outage_timeline.pdf} / \path{uccs_d4b_scan90/plots/outage_story_01/fig_outage_timeline.svg}
  \item 出力(outage内訳CSV):\path{uccs_d4b_scan90/plots/outage_story_01/outage_ranking.csv} / \path{uccs_d4b_scan90/plots/outage_story_01/per_transition.csv}
  \item 出力(尾の寄与分解):\path{uccs_d4b_scan90/plots/pout_tail_01/fig_outage_count_hist.pdf} / \path{uccs_d4b_scan90/plots/pout_tail_01/fig_delta_pout_cum.pdf} / \path{uccs_d4b_scan90/plots/pout_tail_01/pout_tail_decomposition.md}
  \item 出力(条件付きタイミング):\path{uccs_d4b_scan90/plots/ccs_timing_conditional_01/fig_event_triggered_p100_conditional.pdf}
  \item 再現メモ(追加解析):\path{uccs_d4b_scan90/metrics/01/extra_analysis.md}
\end{itemize}

\subsection{実機評価(準備:D4B scan70)}
scan70条件でCCS-offアブレーション(D4B)の追試を行うための実験ディレクトリを用意した.
\begin{itemize}
  \item 実験ルート:\path{uccs_d4b_scan70/}
  \item TX:\path{uccs_d4b_scan70/src/tx/TX_UCCS_D4B_SCAN70/TX_UCCS_D4B_SCAN70.ino}
  \item RX:\path{uccs_d4b_scan70/src/rx/RX_UCCS_D4B_SCAN70/RX_UCCS_D4B_SCAN70.ino}
  \item TXSD:\path{uccs_d4b_scan70/src/txsd/TXSD_UCCS_D4B_SCAN70/TXSD_UCCS_D4B_SCAN70.ino}
  \item truthヘッダ:\path{uccs_d4b_scan70/src/tx/stress_causal_s1_s4_180s.h}
\end{itemize}

\subsection{解析スクリプト(例)}
解析は,1回で完結する巨大スクリプトではなく,「trial集計」「集約」「図表化」に分割して再現可能にする.代表例を以下に示す.
\begin{itemize}
  \item ストレス固定(trial集計):\path{scripts/analyze_stress_causal_real.py}
  \item ストレス固定(集約):\path{scripts/aggregate_stress_causal_real_summary.py}
  \item ストレス固定(v5図生成):\path{scripts/plot_stress_fixed_figures_v5.py}
  \item D2集計(run単体):\path{uccs_d2_scan90/analysis/summarize_d2_run.py}
  \item D2集約(run統合):\path{uccs_d2_scan90/analysis/merge_metrics_runs.py}
  \item D2図生成:\path{uccs_d2_scan90/analysis/plot_power_vs_pout.py}
  \item D4集計(アブレーション):\path{uccs_d4_scan90/analysis/summarize_d4_run_v2.py}
  \item D4図生成:\path{uccs_d4_scan90/analysis/plot_power_vs_pout.py}
  \item D4B集計(CCS-offアブレーション):\path{uccs_d4b_scan90/analysis/summarize_d4b_run_v2.py}
  \item D4B図生成:\path{uccs_d4b_scan90/analysis/plot_power_vs_pout.py}
  \item 統合図生成:\path{uccs_d4b_scan90/analysis/plot_role_separation_overview.py}
  \item $\hat{\rho}_{100}$--$P_{\mathrm{out}}$俯瞰図生成:\path{uccs_d4b_scan90/analysis/plot_alpha_vs_pout.py}
  \item outage事例追跡(D4B run01):\path{uccs_d4b_scan90/analysis/outage_story_trace.py}
  \item 尾の寄与分解(D4B run01):\path{uccs_d4b_scan90/analysis/pout_tail_decomposition.py}
  \item 条件付きタイミング(D4B run01):\path{uccs_d4b_scan90/analysis/ccs_timing_analysis_conditional.py}
  \item ブートストラップCI:\path{scripts/bootstrap_effects.py}
\end{itemize}
