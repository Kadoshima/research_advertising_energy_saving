% chapters/ch9_har_uncertainty.tex --- 第9章 HAR不確実度推定と特徴量
\chapter{HAR不確実度推定と特徴量}

\section{目的}
本研究の制御は,HARの推論結果そのものではなく,「推論がどの程度信頼できるか」を表す不確実度と,「状態がどの程度安定しているか」を表す安定度に基づいて動作する.本章では,端末内推論から不確実度$U$と安定度$S$を得て,複合スコアCCSを構成する流れを整理する.

\section{HAR推論出力と確率分布}
HARモデルは,各時刻$t$においてクラス確率分布$\mathbf{p}(t)=(p_1(t),\dots,p_K(t))$を出力するものとする.本研究では,この確率分布をそのまま通信制御に用いるのではなく,不確実度$U(t)$へ写像する.

\section{不確実度U}
不確実度$U$は,正規化エントロピーとして定義する(第2章参照).温度スケーリング等の較正を行うことで,確率値の過信を抑制し,$U$の意味付けを安定させることが望ましい.

\subsection{較正(Calibration)の位置づけ}
ソフトマックス確率は,ニューラルネットの過信(overconfidence)により,確率値が実際の正解率を過大評価する場合がある.不確実度$U$を通信制御に用いる場合,この過信は「本来短間隔にすべき局面で長間隔を選ぶ」リスクに直結する.そのため,推論結果の確率較正(例:温度スケーリング)により,$U$と実際の誤り確率の対応を改善することが望ましい\scite{guo2017calibration}.

\section{安定度S}
安定度$S$は,直近窓におけるラベル遷移回数や,状態滞在時間に基づいて構成する.例えば,ラベル遷移が少ないほど安定であるという設計意図のもと,$S$を1に近づける.

\subsection{窓長と応答性}
安定度は窓長$W$に依存する.$W$が短いと遷移への応答が速い一方でノイズに敏感になり,$W$が長いと安定する一方で過渡期に追従しにくい.本研究では,実装の単純性を優先し,スライディング窓に基づく遷移回数の正規化という形で$S$を構成する(第2章).

\section{複合スコアCCS}
複合スコアCCSは,確信度$(1-U)$と安定度$S$の線形結合として定義する(第2章参照).係数$\alpha$は設計パラメータであり,実験ではデフォルト値(例:$\alpha=0.7$)を用いつつ,感度は今後の課題として整理する.

\subsection{係数設計と解釈}
係数$\alpha$を大きくすると,瞬時の確信度($1-U$)を重視し,「いま確信があるなら長間隔」という単調な解釈に近づく.一方で$\alpha$を小さくすると,時間的一貫性(安定度$S$)を重視し,短時間の確信度の揺らぎで切替が発生しにくくなる.本研究では,Phase 1の主張(説明容易性)を優先し,確信度を主成分として扱う.

\section{時系列生成とログ}
制御に用いる$U(t)$,$S(t)$,$\mathrm{CCS}(t)$は,スライディング窓で逐次計算し,TXログとして保存する.このログは,オフライン評価(第7章)および実機評価(第4章)で共通に利用するため,ログ形式と時間軸の整合が重要である.

\subsection{計算コストと実装制約}
本研究の想定はバッテリ駆動の小型端末であり,推論・制御・ログの計算コストは制約となる.不確実度$U$は,推論で得られる確率分布からエントロピーを計算するだけであり,追加の推論回数を必要としない.したがって,MC-Dropout等の多回推論を伴う不確実度推定と比較して,計算コストと電力コストが小さい.

\section{まとめ}
本章では,不確実度と安定度からCCSを構成し,通信制御のコンテキストとして利用する設計を整理した.以降の章では,このコンテキストに基づく制御が電力とQoSのトレードオフに与える影響を評価する.
