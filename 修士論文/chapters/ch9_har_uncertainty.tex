% chapters/ch9_har_uncertainty.tex --- 第5章 HAR推論部(TinyML)と不確実度の構成
\section{HAR推論部(TinyML)と不確実度の構成}
\label{sec:har_uncertainty}

\subsection{目的}
本研究の制御は,HARの推論結果そのものではなく,「推論がどの程度信頼できるか」を表す不確実度と,「状態がどの程度安定しているか」を表す安定度に基づいて動作する.本節では,TinyML環境を想定したHAR推論部の位置づけと現状を整理し,端末内推論から不確実度$U$と安定度$S$を得て複合スコアCCSを構成する流れ,および実装上の制約と未完タスクをまとめる.

\subsection{HARモデルと現状}
\subsubsection{入力と出力の概要}
HARは,加速度等の時系列入力を一定窓で切り出し,クラス確率分布を出力する.出力は多クラス(例:12クラス)のまま保持しつつ,通信制御では運用上の粗い状態(例:4クラス)へ集約する場合がある.集約は「制御の行動集合が小さい」「議論を単純化できる」という利点がある一方で,誤認識の影響が集約されるため,評価指標(4クラスBAcc等)を別途監視する必要がある.

\subsubsection{モデル到達点(A0とA\_tiny)}
本研究では,参照モデル(A0)と小型モデル(A\_tiny)を区別する.A0はTFLite int8(PTQ)まで整備されており,TinyML実装の土台となる.一方,A\_tinyは4クラス性能が未達であり,TFLite生成・計測が未完である.表\ref{tab:har_models_status}に,到達点をまとめる.

\begin{table}[tb]
  \centering
  \caption{HARモデルの到達点(例)}
  \label{tab:har_models_status}
  \begin{tabular}{p{0.18\linewidth}p{0.18\linewidth}p{0.18\linewidth}p{0.18\linewidth}p{0.18\linewidth}}
    \toprule
    モデル & 12c BAcc(test) & 4c BAcc(test) & 4c F1(test) & TFLite int8 \\
    \midrule
    A0(参照) & 0.828 & 0.960 & 0.735 & 生成済み(\SI{92.8}{\kilo\byte}) \\
    A\_tiny(現行) & 0.858 & 0.730 & 0.728 & 未生成(未測定) \\
    \bottomrule
  \end{tabular}
\end{table}

数値の根拠は,A0については\texttt{\detokenize{har/004/runs/phase0-1-acc/fold90/metrics.json}},A\_tinyについては\texttt{\detokenize{har/004/runs/phase0-1-acc-tiny/fold90/metrics.json}}に基づく.A0のTFLiteは\texttt{\detokenize{har/004/export/acc_v1_keras/phase0-1-acc.v1.int8.tflite}}であり,sha256は\texttt{\detokenize{e1c3ff0042...}}である(\texttt{\detokenize{har/004/export/acc_v1_keras/manifest.json}}).

\subsubsection{性能要件と未完タスクの位置づけ}
Phase 1の主張は「評価指標と計測系を確立し,ルールベース方策が固定より良い運用点になり得ることを示す」ことである.したがって,HARモデルの最終性能到達はPhase 2に跨る未完タスクとして扱い,本論文では未完であること自体を明示する.これは失敗談ではなく,再現性のために「何が揃っていて,何が揃っていないか」を固定するためである.

\subsection{不確実度U}
HARモデルは,各時刻$t$においてクラス確率分布$\mathbf{p}(t)=(p_1(t),\dots,p_K(t))$を出力する.本研究では,この確率分布から不確実度$U(t)$を導出し,通信制御のコンテキストとする.不確実度$U$は,正規化エントロピーとして定義できる\scite{shannon1948}.
\begin{equation}
  U(t) = -\frac{\sum_{k=1}^{K} p_k(t)\log p_k(t)}{\log K}
\end{equation}
ここで$U=0$は確信度が高い状態,$U=1$は不確実な状態に対応する.$U$は1回の推論出力から計算できるため,多回推論を必要とする不確実度推定(例:MC-Dropout)と比較して計算コストが小さい.

\subsubsection{較正(Calibration)の位置づけ}
ソフトマックス確率は,ニューラルネットの過信(overconfidence)により,確率値が実際の正解率を過大評価する場合がある.不確実度$U$を通信制御に用いる場合,この過信は「本来短間隔にすべき局面で長間隔を選ぶ」リスクに直結する.そのため,推論結果の確率較正(例:温度スケーリング)により,$U$と実際の誤り確率の対応を改善することが望ましい\scite{guo2017calibration}.

\subsection{安定度S}
安定度$S$は,直近窓におけるラベル遷移回数や,状態滞在時間に基づいて構成する.例えば,ラベル遷移が少ないほど安定であるという設計意図のもと,$S$を1に近づける.

\subsubsection{窓長と応答性}
安定度は窓長$W$に依存する.$W$が短いと遷移への応答が速い一方でノイズに敏感になり,$W$が長いと安定する一方で過渡期に追従しにくい.本研究では,実装の単純性を優先し,スライディング窓に基づく遷移回数の正規化という形で$S$を構成する(第2章).

\subsection{複合スコアCCS}
複合スコアCCSは,確信度$(1-U)$と安定度$S$の線形結合として定義する.
\begin{equation}
  \mathrm{CCS}(t)=\alpha\,(1-U(t))+(1-\alpha)\,S(t)
\end{equation}
係数$\alpha$は設計パラメータであり,Phase 1では説明容易性を優先して固定値(例:$\alpha=0.7$)を採用する.ただし,$\alpha$や窓長$W$が変わるとCCS分布が変化し,閾値の意味が変わるため,モデル更新時には閾値再キャリブレーションが必要となる.

\subsubsection{係数設計と解釈}
係数$\alpha$を大きくすると,瞬時の確信度($1-U$)を重視し,「いま確信があるなら長間隔」という単調な解釈に近づく.一方で$\alpha$を小さくすると,時間的一貫性(安定度$S$)を重視し,短時間の確信度の揺らぎで切替が発生しにくくなる.本研究では,Phase 1の主張(説明容易性)を優先し,確信度を主成分として扱う.

\subsection{時系列生成とログ}
制御に用いる$U(t)$,$S(t)$,$\mathrm{CCS}(t)$は,スライディング窓で逐次計算し,TXログとして保存する.このログは,オフライン評価および実機評価で共通に利用するため,ログ形式と時間軸の整合が重要である.

\subsubsection{計算コストと実装制約}
本研究の想定はバッテリ駆動の小型端末であり,推論・制御・ログの計算コストは制約となる.不確実度$U$は,推論で得られる確率分布からエントロピーを計算するだけであり,追加の推論回数を必要としない.したがって,MC-Dropout等の多回推論を伴う不確実度推定と比較して,計算コストと電力コストが小さい.

\subsection{量子化(PTQ int8)とTFLite整合(参照モデル)}
TinyML環境では,推論コストとメモリを抑えるため,TFLite int8(PTQ)等の量子化が重要となる.A0(参照モデル)では,TFLite int8を生成し,PyTorch出力との整合(argmax一致率,max\_probのMAE)を確認している.これにより,「推論値の揺らぎ」が制御の閾値に与える影響を監査できる.

\begin{table}[tb]
  \centering
  \caption{TFLiteアーティファクト(参照モデルA0,例)}
  \label{tab:tflite_artifact_a0}
  \begin{tabular}{p{0.28\linewidth}p{0.62\linewidth}}
    \toprule
    項目 & 値 \\
    \midrule
    TFLite int8 & \texttt{\detokenize{har/004/export/acc_v1_keras/phase0-1-acc.v1.int8.tflite}} \\
    sha256 & \texttt{\detokenize{e1c3ff0042badac5dd9bb478a17fac79991736c813143b45e5cb981d9db68610}} \\
    サイズ & \SI{92.8}{\kilo\byte} \\
    代表データ & \texttt{\detokenize{har/001/export/acc_v1/rep_data.npy}}(rep\_samples=2000) \\
    PT vs TFLite & argmax一致=0.98,max\_prob MAE=0.0277(200サンプル) \\
    \bottomrule
  \end{tabular}
\end{table}

\subsection{未完タスク(A\_tinyの実機導入に向けて)}
本論文執筆時点で,A\_tinyについては以下が未完である.
\begin{itemize}
  \item 4クラス性能の回復(BAcc\,$\ge0.80$等の要件を満たすまでの改善).
  \item TFLite生成(PTQ int8)とサイズ・整合の計測(PT↔TFLite誤差,sha256記録).
  \item ESP32上での推論時間$t_{\mathrm{inf}}$とArenaサイズの計測.
  \item CCS閾値($\theta_{\mathrm{low/high}}$)の再キャリブレーション(TFLite出力ベース).
\end{itemize}
これらは,単に精度改善のためだけではなく,「閾値で制御する際の意味づけを安定させる」ために必要である.本研究は,これらを将来課題として明示し,評価系(電力+QoSの同時測定)がそのままPhase 2へ接続できる形で整備されている点を主張する.

\subsection{まとめ}
本節では,TinyML環境を想定したHAR推論部の現状と,不確実度$U$・安定度$S$・CCSの構成を整理した.また,量子化とTFLite整合の重要性,および未完タスクを明示した.以降の章では,このコンテキストに基づく広告間隔制御と,計測・指標に基づく評価を述べる.
