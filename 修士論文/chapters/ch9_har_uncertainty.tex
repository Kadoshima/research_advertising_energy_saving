% chapters/ch9_har_uncertainty.tex --- 第5章 HAR推論部(TinyML)と不確実度の構成
\section{HAR推論部(TinyML)と不確実度の構成}
\label{sec:har_uncertainty}

\subsection{目的}
本研究の制御は,HARの推論結果そのものではなく,「推論がどの程度信頼できるか」を表す不確実度と,「状態がどの程度安定しているか」を表す安定度に基づいて動作する.本節では,TinyML環境を想定したHAR推論部の位置づけと現状を整理し,端末内推論から不確実度$U$と安定度$S$を得て複合スコアCCSを構成する流れ,および実装上の制約と未完タスクをまとめる.

\subsection{HARモデルと現状}
HARは,加速度等の時系列入力を一定窓で切り出し,クラス確率分布を出力する.本研究では参照モデル(A0)と小型モデル(A\_tiny)を区別し,Phase 1では評価指標・計測系・制御の成立を優先するため,最終的なTinyML最適化(A\_tinyの性能到達やTFLite整合の監査)はPhase 2に跨る将来課題として位置づける.

\subsection{不確実度U}
HARモデルは,各時刻$t$においてクラス確率分布$\mathbf{p}(t)=(p_1(t),\dots,p_K(t))$を出力する.本研究では,この確率分布から不確実度$U(t)$を導出し,通信制御のコンテキストとする.不確実度$U$は,正規化エントロピーとして定義できる\scite{shannon1948}.
\begin{equation}
  U(t) = -\frac{\sum_{k=1}^{K} p_k(t)\log p_k(t)}{\log K}
\end{equation}
ここで$U=0$は確信度が高い状態,$U=1$は不確実な状態に対応する.$U$は1回の推論出力から計算できるため,多回推論を必要とする不確実度推定(例:MC-Dropout)と比較して計算コストが小さい.

\subsubsection{較正(Calibration)の位置づけ}
ソフトマックス確率は,ニューラルネットの過信(overconfidence)により,確率値が実際の正解率を過大評価する場合がある.不確実度$U$を通信制御に用いる場合,この過信は「本来短間隔にすべき局面で長間隔を選ぶ」リスクに直結する.そのため,推論結果の確率較正(例:温度スケーリング)により,$U$と実際の誤り確率の対応を改善することが望ましい\scite{guo2017calibration}.

\subsection{安定度S}
安定度$S$は,直近窓におけるラベル遷移回数や,状態滞在時間に基づいて構成する.例えば,ラベル遷移が少ないほど安定であるという設計意図のもと,$S$を1に近づける.

\subsubsection{窓長と応答性}
安定度は窓長$W$に依存する.$W$が短いと遷移への応答が速い一方でノイズに敏感になり,$W$が長いと安定する一方で過渡期に追従しにくい.本研究では,実装の単純性を優先し,スライディング窓に基づく遷移回数の正規化という形で$S$を構成する(第2章).

\subsection{複合スコアCCS}
複合スコアCCSは,確信度$(1-U)$と安定度$S$の線形結合として定義する.
\begin{equation}
  \mathrm{CCS}(t)=\alpha\,(1-U(t))+(1-\alpha)\,S(t)
\end{equation}
係数$\alpha$は設計パラメータであり,Phase 1では説明容易性を優先して固定値(例:$\alpha=0.7$)を採用する.ただし,$\alpha$や窓長$W$が変わるとCCS分布が変化し,閾値の意味が変わるため,モデル更新時には閾値再キャリブレーションが必要となる.

\subsubsection{係数設計と解釈}
係数$\alpha$を大きくすると,瞬時の確信度($1-U$)を重視し,「いま確信があるなら長間隔」という単調な解釈に近づく.一方で$\alpha$を小さくすると,時間的一貫性(安定度$S$)を重視し,短時間の確信度の揺らぎで切替が発生しにくくなる.本研究では,Phase 1の主張(説明容易性)を優先し,確信度を主成分として扱う.

\subsection{時系列生成とログ}
制御に用いる$U(t)$,$S(t)$,$\mathrm{CCS}(t)$は,スライディング窓で逐次計算し,TXログとして保存する.このログは,オフライン評価および実機評価で共通に利用するため,ログ形式と時間軸の整合が重要である.

本研究の想定はバッテリ駆動の小型端末であり,推論・制御・ログの計算コストは制約となる.不確実度$U$は推論出力から1回で計算できるため,多回推論を要する不確実度推定より計算コストが小さい.一方で,量子化や確率較正により$U$と閾値の意味づけが変化し得るため,モデル更新時は閾値再キャリブレーションが必要となる.

\subsection{まとめ}
本節では,TinyML環境を想定したHAR推論部の現状と,不確実度$U$・安定度$S$・CCSの構成を整理した.また,量子化とTFLite整合の重要性,および未完タスクを明示した.以降の章では,このコンテキストに基づく広告間隔制御と,計測・指標に基づく評価を述べる.
