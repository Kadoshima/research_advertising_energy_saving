% chapters/ch3_system_design.tex --- 第4章 問題設定とシステム設計
\section{問題設定とシステム設計}

\subsection{World Model}
本研究は,アプリケーション層(HAR)と通信層(BLE広告)を跨ぐクロスレイヤー制御である.このため,単に広告間隔を調整するだけでなく,「どの情報がどこで生成され,どこで観測されるか」を明確にする必要がある.本節では,役者(Actors)と能力(Capabilities)を整理し,議論の前提を固定する.

\subsubsection{Actors}
表\ref{tab:actors}に,本研究で扱う役者を示す.

\begin{table}[tb]
  \centering
  \caption{Actors(役者)}
  \label{tab:actors}
  \begin{tabular}{p{0.28\linewidth}p{0.62\linewidth}}
    \toprule
    Actor & 役割 \\
    \midrule
    Edge Node / DUT & IMU等からHARを推論し,不確実度$U$・安定度$S$・CCSを生成して広告間隔を制御する(送信側) \\
    Gateway(スマートフォン) & BLE広告を受信する主体.スキャン挙動はOS依存で非理想とみなす(受信側) \\
    Observer(計測系) & 電力(TXSD)と受信(RX)を収集し,一次KPIを算出する(実験時の観測者) \\
    Safety Oracle(概念) & QoS制約$P_{\mathrm{out}}(\tau)\le\delta$を規定する境界(設計上の規格) \\
    Policy Engine & コンテキストから広告間隔を決定する意思決定器(Phase 1はルール,Phase 2でSafe Bandit) \\
    \bottomrule
  \end{tabular}
\end{table}

\subsubsection{Capabilities(5つの柱)}
本研究の設計で重要となる能力を以下にまとめる.
\begin{enumerate}
  \item Perception:HAR推論から確率分布を得て,不確実度$U$と安定度$S$を計算する.
  \item Decision:$U,S$からCCSを構成し,広告間隔$a$を決定する.
  \item Communication:BLE広告として送信し,受信側の非理想スキャン下で到達する.
  \item Safety:QoS制約を守るため,危険兆候では短間隔へ退避できる.
  \item Observability:意思決定の根拠($U,S,\mathrm{CCS}$,閾値,切替理由)をログに残す.
\end{enumerate}

\subsection{一次KPIと不変条件}
本研究では,議論がぶれないように一次KPIを固定する.表\ref{tab:kpi}に一次・二次KPIを示す.一次KPIは「省電力」と「期限超過」の同時評価を担い,二次KPIは補助指標として扱う.

\begin{table}[tb]
  \centering
  \caption{KPIの整理}
  \label{tab:kpi}
  \begin{tabular}{p{0.22\linewidth}p{0.68\linewidth}}
    \toprule
    区分 & 指標 \\
    \midrule
    一次(Primary) & $\si{\micro\coulomb}/\text{event}$,$P_{\mathrm{out}}(\tau)$,TL\_p95 \\
    二次(Secondary) & 平均電力$\overline{P}$,PDR(unique),RSSI,切替頻度(switch\_rate),平均広告レート(adv\_rate) \\
    \bottomrule
  \end{tabular}
\end{table}

また,設計上の不変条件として以下を採用する.
\begin{itemize}
  \item 後方互換のIF:\texttt{decide\_interval(context) -> interval} の構造を保ち,ルールから学習へ差し替え可能にする.
  \item 端末内比較:受信端末・設定を固定し,OS依存の影響は端末内で相殺して解釈する.
  \item ログ優先:KPI算出に必要な列を必ず残し,派生値は後処理で再計算できる形にする.
\end{itemize}

\subsection{目的関数と制約}
本研究の問題設定は,広告間隔制御により電力(電荷消費)を下げつつ,期限超過率を制約することである.動的制御の方策$\pi$に対し,
\begin{equation}
  \text{minimize}\quad \mathbb{E}[q_{\mathrm{event}}(\pi)]
  \quad \text{subject to}\quad P_{\mathrm{out}}(\tau;\pi)\le \delta
\end{equation}
として表せる.ここで$q_{\mathrm{event}}$(単位:$\si{\micro\coulomb}/\text{event}$)は,セッション中の電荷消費をイベント数で正規化した指標である(定義は\secref{sec:metrics_detail}).コンテキストとしては$U,S$(またはCCS)を用い,Phase 1ではルール$\pi_{\mathrm{rule}}$,Phase 2では学習$\pi_{\mathrm{safe\_mab}}$を想定する.

\subsection{フェイルセーフと運用設計}
非理想スキャン環境では,受信品質が瞬間的に悪化する可能性がある.このため,実装では安全側への退避(短間隔)を設けることが望ましい.本研究では,切替規則の設計とログ化により,安全側に寄せた挙動を説明可能にする.
具体的には,通常はCCSに応じて広告間隔を選択し,QoS悪化の兆候がある場合は短間隔へ退避できる余地を残す.また,退避の根拠(切替理由)をログに残し,監査可能にする.

\subsection{フェーズ設計(Phase 1 → Phase 2)}
Phase 1では,ルールベース方策で「計測・指標・実機実装」を成立させることを優先する.特に,TL/Poutの定義と時間同期,電力計測の健全化は,学習以前に必須である.Phase 2では,コンテキスト($U,S$)と行動(広告間隔)の選択をSafe Contextual Banditとして定式化し,環境差に適応しつつQoS制約を守る枠組みを検証する.
