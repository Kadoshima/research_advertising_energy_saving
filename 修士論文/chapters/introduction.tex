% chapters/introduction.tex --- はじめに(番号なし、目次には載せる)
\chapterwithtoc{はじめに}

ウェアラブル端末やエッジデバイスにおいて,推論結果を近傍のスマートフォンへ低遅延に通知する仕組みは,見守りや労働安全など多くの応用で重要である.一方で,常時高頻度に通信を行うと電力消費が増大し,小型バッテリでの長時間運用が困難となる.Bluetooth Low Energy(BLE)の広告(Advertising)は接続を前提としない軽量な近傍通信手段として広く利用されているが,受信側(スマートフォン)のスキャン動作はOSや端末状態に依存し,連続スキャンが保証されない.したがって,広告間隔を固定値として設計した場合,環境や状態変化に対して最適性・安全性の両面で限界がある.

本研究は,アプリケーション層(Human Activity Recognition:HAR)で得られる推論不確実度を通信層の制御に利用し,QoS制約(一定時間以内に受信されること)を満たしながら省電力化する枠組みを扱う.まず,推論不確実度と時間的安定度から複合スコアを構成し,その値に応じて広告間隔を段階的に切り替えるルールベース方策を定義する.次に,送信・受信・電力計測を統合した計測基盤を整備し,平均電力と受信遅延(Time-to-first-Receive:TL)および期限超過率(Outage Probability:$P_{\mathrm{out}}(\tau)$)を同時に評価できる形に整理する.

本論文の構成を以下に示す.第1章では研究背景と関連研究を述べる.第2章では問題設定と提案手法を示す.第3章では計測システムと評価指標,実験条件を説明する.第4章では固定間隔および動的切替の評価結果を示す.第5章では結果の考察と限界,今後の課題を述べる.第6章ではストレス固定実験を用いた指標定義(v5)と時間同期の手順を整理する.第7章ではmHealth合成に基づくオフライン評価と代表方策の選定を述べる.第8章では計測系の健全化と再現性確保の観点をまとめる.第9章ではHAR不確実度推定と特徴量を整理する.第10章では実装(ファームウェア・解析パイプライン)の要点を述べる.最後におわりにまとめ,付録で指標定義や追加図表を補足する.
