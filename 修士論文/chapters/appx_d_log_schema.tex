% chapters/appx_d_log_schema.tex --- 付録D ログスキーマ(詳細)
\section{ログスキーマ(詳細)}
\label{sec:log_schema}

本節では,本研究で扱うログ(TXSD/RX/TX)の列定義をまとめる.解析スクリプトは列追加に対して後方互換で拡張可能であるが,指標計算に必須な列は固定する.

\subsection{TXSD(電力)ログ}
TXSDログは,電流・電圧系列と,末尾のsummaryから構成される.生系列は後処理で再積分できるように保持し,summaryはquick checkとして利用する.

\begin{longtable}{p{0.22\linewidth}p{0.18\linewidth}p{0.52\linewidth}}
  \caption{TXSDログ(例)の列定義}\label{tab:txsd_schema}\\
  \toprule
  列 & 単位 & 意味 \\
  \midrule
  \endfirsthead
  \toprule
  列 & 単位 & 意味 \\
  \midrule
  \endhead
  ms & ms & 試行開始からの相対時刻 \\
  mv & mV & 計測電圧(整数) \\
  uA & $\mu$A & 計測電流(整数) \\
  p\_mW & mW & 瞬時電力(派生列,後処理で再計算可能) \\
  \midrule
  \multicolumn{3}{l}{summary(末尾に出力する集約値の例)} \\
  \midrule
  ms\_total & ms & 試行時間 \\
  adv\_count & 回 & 広告イベント数(TICK物理カウント等) \\
  E\_total\_mJ & mJ & 総エネルギー(積分値) \\
  avg\_power\_mW & mW & 平均電力($E/T$) \\
  \bottomrule
\end{longtable}

\subsection{RX(受信)ログ}
RXログは,受信時刻とペイロード識別子(seq等)を保持する.PDR\_unique,TL,$P_{\mathrm{out}}(\tau)$の算出に必須である.

\begin{longtable}{p{0.22\linewidth}p{0.18\linewidth}p{0.52\linewidth}}
  \caption{RXログ(例)の列定義}\label{tab:rx_schema}\\
  \toprule
  列 & 単位 & 意味 \\
  \midrule
  \endfirsthead
  \toprule
  列 & 単位 & 意味 \\
  \midrule
  \endhead
  t\_ms & ms & 受信時刻(試行開始からの相対,または端末時刻) \\
  seq & --- & ペイロードに埋め込んだ識別子(広告イベント由来) \\
  rssi & dBm & 受信RSSI \\
  dedup\_flag & --- & 重複除外のためのフラグ(ある場合) \\
  \bottomrule
\end{longtable}

\subsection{TX(送信)ログ}
TXログは,方策の状態($U,S,\mathrm{CCS},a$等)を保持する.オフライン評価と実機評価を接続するため,方策の根拠(切替理由)を残すことが望ましい.

\begin{longtable}{p{0.22\linewidth}p{0.18\linewidth}p{0.52\linewidth}}
  \caption{TXログ(例)の列定義}\label{tab:tx_schema}\\
  \toprule
  列 & 単位 & 意味 \\
  \midrule
  \endfirsthead
  \toprule
  列 & 単位 & 意味 \\
  \midrule
  \endhead
  t\_ms & ms & 時刻 \\
  U & --- & 不確実度 \\
  S & --- & 安定度 \\
  CCS & --- & 複合スコア \\
  adv\_interval\_ms & ms & 選択した広告間隔 \\
  reason & --- & 切替理由(閾値越え等) \\
  \bottomrule
\end{longtable}

\subsection{ファイル命名と配置}
ログは,試行(trial)単位でファイルを分割し,\texttt{\detokenize{trial_XXX.csv}}等の連番で管理する.このとき,条件IDや実験日付をパスまたはファイル名に含め,作業ログに生成コマンドと出力先を残す(\secref{sec:runbook}).
