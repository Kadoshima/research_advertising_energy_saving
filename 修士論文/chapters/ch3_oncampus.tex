% chapters/ch3_oncampus.tex --- 第3章 評価方法
\chapter{評価方法}

\section{計測構成}
本研究では,送信(TX)・電力ロガ(TXSD)・受信(RX)の三ノード構成でログを取得する.各ノードの役割,配線,同期信号,およびログの保存形式を統一し,条件間の比較が可能な形に整理する.

\begin{table}[tb]
  \centering
  \caption{評価系のノード構成}
  \label{tab:nodes}
  \begin{tabular}{lll}
    \toprule
    ノード & 役割 & 主なログ \\
    \midrule
    TX & BLE広告送信・間隔制御 & ペイロード(時刻/タグ),制御状態 \\
    TXSD & 電力計測(INA219) & 電圧・電流系列,集約指標 \\
    RX & 受信ログ取得 & 受信時刻,RSSI,ペイロード \\
    \bottomrule
  \end{tabular}
\end{table}

\subsection{機材と計測パラメータ}
本研究の計測は,汎用マイコン(ESP32系)と電流センサ(INA219)を中心に構成する.表\ref{tab:hw}に,主要な構成要素と役割を示す.

\begin{table}[tb]
  \centering
  \caption{計測システムの主要構成}
  \label{tab:hw}
  \begin{tabular}{ll}
    \toprule
    要素 & 役割 \\
    \midrule
    ESP32(TX) & BLE広告送信,広告間隔制御,ペイロード生成 \\
    ESP32(TXSD) & INA219計測データの収集,SD保存,エネルギー集約 \\
    ESP32(RX)/スマートフォン & 受信ログの取得(時刻,RSSI,seq) \\
    INA219 & 電流・電圧の計測(サンプリング周期は試行条件に依存) \\
    SYNC/TICK & 試行区間の同期,広告回数の物理カウント \\
    \bottomrule
  \end{tabular}
\end{table}

\subsection{同期信号}
試行(trial)の開始・終了を揃えるために,同期信号(SYNC)を用いる.TXは試行区間の境界でSYNC信号を出力し,TXSDおよびRXはこの信号に同期してログ取得区間を区切る.

\subsection{広告回数のカウント}
広告イベント数は,到達性指標の分母として重要である.可能であれば,TXがTICK信号を出力し,TXSD側で割り込みにより広告回数を物理カウントする.TICKが利用できない場合は,試行時間と広告間隔から期待回数を推定する.

\section{ログスキーマ}
解析の再現性を担保するため,ログの列(カラム)を固定する.本研究では,TXSDログ末尾のsummary(例:$N_{\mathrm{adv}}$,$E$,$\overline{P}$)を一次情報として扱い,RXログはPDR/TL評価の基礎とする.

\subsection{TXSD(電力)ログ}
TXSDログは,時刻と計測値(電圧・電流)を行ごとに保持し,末尾に集約値を出力する.積分は$\Delta t$を用いた逐次積分とし,後処理でも再積分して監査する(第8章).

\subsection{RX(受信)ログ}
RXログは,受信時刻,RSSI,ペイロード(seq等)を保持する.重複受信があるため,seqでユニーク化した件数を併記する(第6章・付録A).

\section{評価指標}
本研究で用いる主要指標を以下に示す.
\begin{itemize}
  \item PDR:広告イベントに対する受信の割合(重複受信を考慮した指標を併記する).
  \item TL:状態遷移(イベント)後に最初に受信されるまでの遅延.
  \item $P_{\mathrm{out}}(\tau)$:期限$\tau$以内に受信できない確率.
  \item 平均電力:TXSDログから算出する平均消費電力.
\end{itemize}
指標定義と計算上の注意(時間同期など)は,以降の章で結果の解釈に直結するため,実装と整合する形で固定する.

\subsection{PDR}
PDRは受信側ログの件数を単純に数えるだけでは,重複受信により1を超える場合がある.したがって,seq等の識別子でユニーク化した件数に基づく指標も併記し,QoS用途にはユニーク化した指標を優先する.

\subsection{TLと時間同期}
TLと$P_{\mathrm{out}}(\tau)$の算出には,「イベント時刻」と「受信ログの時間軸」が一致している必要がある.実機では開始タイミングのずれが入り得るため,seqと広告間隔から期待時刻を復元し,定数オフセットとして補正した時刻でTLを算出する.

\subsection{平均電力}
平均電力は,TXSDログから得られる総エネルギーと試行時間から算出する.サンプリング周期が異なる場合でも,dtを正しく積分できていれば総エネルギーは整合するため,試行間比較が可能である.

\section{実験条件}
実験は,固定間隔(例:\SI{100}{\milli\second},\SI{500}{\milli\second})と,動的切替(2値制御)を含む複数条件で実施する.環境条件(距離,干渉状況,受信側スキャン設定)を記録し,再現性を担保する.

\subsection{端末内比較の原則}
受信側の非理想スキャンは端末・OS状態に依存するため,本研究の比較は原則として端末内比較(同一端末・同一設定)として行う.また,試行ごとにスキャンモード,画面状態,電源状態等を記録し,評価の前提を明示する.

\subsection{固定間隔条件}
固定間隔条件では,広告間隔を所定値(例:\SI{100}{\milli\second},\SI{500}{\milli\second})に固定し,基準となる平均電力と受信品質を取得する.この基準値は,オフライン評価における電力テーブルおよびQoS曲線の土台として用いる.

\subsection{動的切替条件}
動的切替条件では,2値制御(\SI{100}{\milli\second}/\SI{500}{\milli\second})を用い,固定条件との比較を行う.試行内で切替が発生するため,ペイロードに切替状態を埋め込み,RXログおよびTXSDログと整合する形で解析する.

\section{オフライン評価}
実機実験の補助として,ログに基づくオフライン評価を行う.具体的には,固定間隔の電力テーブルと,受信品質($P_{\mathrm{out}}(\tau)$)の推定を組み合わせ,QoS制約下での省電力運用点を探索する.

\subsection{目的}
オフライン評価の目的は,実機試験の探索空間を縮小し,説明可能な代表方策を選定することである.特に,QoS制約$\delta$の境界付近で成立する運用点を把握し,固定間隔に対してどの程度省電力になり得るかを見積もる.

\subsection{手順}
オフライン評価は,概ね以下の手順で実施する.
\begin{enumerate}
  \item 固定間隔の実測から,広告間隔ごとの平均電力(電力テーブル)を作成する.
  \item 受信品質の基準データから,固定間隔における$P_{\mathrm{out}}(\tau)$等を推定する.
  \item 不確実度・安定度の時系列(ログまたは合成データ)に対して,候補方策(閾値・ヒステリシス等)を適用し,間隔の滞在比率を算出する.
  \item 滞在比率と電力テーブルを合成して平均電力を推定し,QoS制約を満たす点の集合からParetoフロント等を抽出する.
\end{enumerate}

\section{データ処理の概要}
本研究の解析は,ログを読み込んで指標を算出し,条件ごとに集約して図表を生成する流れである.最小構成として,以下を満たすことを解析の合格条件とする.
\begin{itemize}
  \item 試行区間がSYNCで正しく切れている(期間$T$が想定と一致する).
  \item TXSDの集約値と後処理の再積分が同程度である(単位・欠損が健全).
  \item RXのPDR\_uniqueが0付近にならない(受信系が動作している).
  \item TL/Poutの算出に時間同期(定数オフセット補正)を適用できる.
\end{itemize}
