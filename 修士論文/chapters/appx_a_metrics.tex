% chapters/appx_a_metrics.tex --- 付録A 指標定義(詳細)
\chapter{指標定義(詳細)}

本付録では,本研究で用いる主要指標の定義をまとめる.本文では要点のみ述べ,詳細はここに集約する.

\section{PDR}
PDRは,広告イベント数に対して受信できた割合である.重複受信がある場合,単純な受信行数に基づくPDRは1を超える可能性があるため,seq等でユニーク化した指標も併記する.

\subsection{定義}
本研究では,広告イベント数(分母)を$N_{\mathrm{adv}}$,受信ログ行数(重複含む)を$N_{\mathrm{rx}}$,seqでユニーク化した受信数を$N_{\mathrm{rx,uniq}}$とする.
\begin{equation}
  \mathrm{PDR}_{\mathrm{raw}} = \frac{N_{\mathrm{rx}}}{N_{\mathrm{adv}}},\qquad
  \mathrm{PDR}_{\mathrm{unique}} = \frac{N_{\mathrm{rx,uniq}}}{N_{\mathrm{adv}}}
\end{equation}
$\mathrm{PDR}_{\mathrm{raw}}$は重複受信により1を超え得るため,到達性の比較(QoS用途)では$\mathrm{PDR}_{\mathrm{unique}}$を優先する.

\subsection{分母(広告イベント数)の扱い}
広告イベント数$N_{\mathrm{adv}}$は,可能な場合はTICKにより物理カウントした値を正とする.TICKが利用できない場合は,試行時間と広告間隔から期待回数を推定するが,動的切替では誤差が入り得るため,解釈には注意する.

\section{TLとPout(τ)}
TLは,イベント後に最初に受信されるまでの遅延である.Pout(τ)は,TLが期限$\tau$を超える確率として定義する.TL/Poutの算出では,受信ログの時間軸と真値(truth)の時間軸の一致が必要であり,開始位相ずれがある場合は定数オフセットで補正する(第6章).

\subsection{イベントと遅延}
真値(truth)における状態遷移(イベント)時刻を$t_{\mathrm{event},j}$,その後に初めて正しいラベル(または該当イベントに対応する識別子)が受信される時刻を$t_{\mathrm{rx},j}$とすると,
\begin{equation}
  \mathrm{TL}_j = t_{\mathrm{rx},j} - t_{\mathrm{event},j}
\end{equation}
である.期限$\tau$に対する期限超過率は,
\begin{equation}
  P_{\mathrm{out}}(\tau)=\frac{1}{N_{\mathrm{event}}}\sum_{j=1}^{N_{\mathrm{event}}}\mathbb{I}[\mathrm{TL}_j>\tau]
\end{equation}
で定義する.

\subsection{時間同期(定数オフセット補正)}
実機ログでは,RXログの開始が遅れる等の理由で,RXログの時間軸とtruthの時間軸が定数だけずれている場合がある.固定間隔リプレイ(seqが単調増加)を前提に,次の定数オフセットで補正する.広告間隔を$\Delta t$(ms),RXログにおけるseqの初回観測時刻を$\mathrm{first\_ms}(\mathrm{seq})$とすると,
\begin{equation}
  \mathrm{offset\_ms} = \mathrm{median}_{\mathrm{seq}>0}\left(\mathrm{seq}\cdot\Delta t - \mathrm{first\_ms}(\mathrm{seq})\right)
\end{equation}
として推定し,$\mathrm{ms\_aligned}=\mathrm{ms}+\mathrm{offset\_ms}$で補正した時刻を用いてTL/Poutを計算する(第6章).

\subsection{末端遷移と有効長}
試行時間がtruthの全長より短い場合,末端の遷移が試行区間外に出る.この場合,評価対象のイベント数が試行ごとに変化し,Poutの分母が揺れる.本研究では,TX側のクランプ長と同様にtruth側も有効長(EFFECTIVE\_LEN)でクリップし,イベント抽出範囲を揃える.

\section{平均電力とエネルギー}
平均電力は,試行期間における総エネルギーを時間で割ることで算出する.ログの生データ(mV/µA)から再積分した値と,summaryの整合を監査することで,単位換算や欠損による誤差を抑制する.

\subsection{逐次積分}
TXSDログで得られる電圧$V(t)$と電流$I(t)$から瞬時電力$P(t)=V(t)I(t)$を算出し,サンプリング間隔$\Delta t$で逐次積分することで総エネルギー$E$を求める.実装では,単位の一貫性(mV,$\mu$A,ms等)を保つため,生ログから後処理で再計算できる形にし,summaryとの差分を監査する.

\subsection{差分指標(ON-OFF)}
計測系の定常負荷が大きい場合,広告間隔による差分が平均電力に埋もれる.そのため,広告ONとOFFの差分$\Delta E=E_{\mathrm{on}}-E_{\mathrm{off}}$を広告イベント数$N_{\mathrm{adv}}$で正規化した$\Delta E/N_{\mathrm{adv}}$を併用し,「無線1回当たりの増分」として評価する.
