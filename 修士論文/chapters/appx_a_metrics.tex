% chapters/appx_a_metrics.tex --- 付録A 指標定義(詳細)
\section{指標定義(詳細)}
\label{sec:metrics_detail}

本節では,本研究で用いる主要指標を,実装と集計の観点で最小限に定義する.本文の図表は本節の定義に従い,同一の処理パイプラインで端末内比較を行う.

\subsection{PDR(到達率)}
PDRは,広告イベント数に対して受信できた割合である.重複受信がある場合,受信行数に基づくPDRは1を超え得るため,seq等でユニーク化した指標も併記する.

本研究では,広告イベント数(分母)を$N_{\mathrm{adv}}$,受信ログ行数(重複含む)を$N_{\mathrm{rx}}$,seqでユニーク化した受信数を$N_{\mathrm{rx,uniq}}$とする.
\begin{equation}
  \mathrm{PDR}_{\mathrm{raw}} = \frac{N_{\mathrm{rx}}}{N_{\mathrm{adv}}},\qquad
  \mathrm{PDR}_{\mathrm{unique}} = \frac{N_{\mathrm{rx,uniq}}}{N_{\mathrm{adv}}}
\end{equation}
$\mathrm{PDR}_{\mathrm{raw}}$は重複受信により1を超え得るため,到達性の比較(QoS用途)では$\mathrm{PDR}_{\mathrm{unique}}$を優先する.分母$N_{\mathrm{adv}}$は,可能な場合はTICKにより物理カウントした値を正とし,利用できない場合は試行時間と広告間隔から推定する(動的切替では誤差が入り得る).

\subsection{TLとPout(τ)}
TLは,イベント後に最初に受信されるまでの遅延である.Pout(τ)は,TLが期限$\tau$を超える確率として定義する.TL/Poutの算出では,受信ログの時間軸と真値(truth)の時間軸の一致が必要であり,開始位相ずれがある場合は定数オフセットで補正する(\secref{sec:stress_fixed_metrics}).

真値(truth)における状態遷移(イベント)時刻を$t_{\mathrm{event},j}$,その後に初めて正しいラベル(または該当イベントに対応する識別子)が受信される時刻を$t_{\mathrm{rx},j}$とすると,
\begin{equation}
  \mathrm{TL}_j = t_{\mathrm{rx},j} - t_{\mathrm{event},j}
\end{equation}
である.期限$\tau$に対する期限超過率は,
\begin{equation}
  P_{\mathrm{out}}(\tau)=\frac{1}{N_{\mathrm{event}}}\sum_{j=1}^{N_{\mathrm{event}}}\mathbb{I}[\mathrm{TL}_j>\tau]
\end{equation}
で定義する.
\noindent
また,試行時間がtruthの全長より短い場合は末端遷移が試行区間外に出るため,truth側も有効長(EFFECTIVE\_LEN)でクリップし,$N_{\mathrm{event}}$の分母を揃える.

\subsection{電荷と$q_{\mathrm{event}}$($\si{\micro\coulomb}/\text{event}$)}
本研究の一次KPIである$q_{\mathrm{event}}$は,セッション中の電荷消費をイベント数で正規化した指標である.電力やエネルギーは電圧に依存するため,バッテリ消費の観点では電荷(Coulomb count)で整理すると解釈が容易である.

電流を$I(t)$とすると,総電荷$Q$は
\begin{equation}
  Q=\int I(t)\,dt
\end{equation}
で定義できる.TXSDログでは電流が$\mu$A,時刻がmsで記録されるため,離散近似として
\begin{equation}
  Q_{\si{\micro\coulomb}} \approx \sum_{i} I_{\si{\micro\ampere},i}\cdot \frac{\Delta t_{\si{\milli\second},i}}{1000}
\end{equation}
により$\si{\micro\coulomb}$単位の電荷を算出できる.

計測系の定常負荷(ロギング,LED等)が大きい場合,広告間隔による差分が総電荷に埋もれる.このため,広告ONとOFFの差分
\begin{equation}
  \Delta Q = Q_{\mathrm{on}}-Q_{\mathrm{off}}
\end{equation}
を用いて,広告による増分を評価する.OFF計測では,sleep状態や無線スタック状態がONと一致するよう,同一コード系列で条件を固定する必要がある(\secref{sec:runbook}).

TL/Poutの評価起点となるイベント数を$N_{\mathrm{event}}$とすると,
\begin{equation}
  q_{\mathrm{event}}=\frac{\Delta Q}{N_{\mathrm{event}}}
\end{equation}
で定義する.$\Delta Q$の代わりに$Q_{\mathrm{on}}$を用いる場合は,基準補正が入らないため,条件間比較の解釈に注意する.
\noindent
なお,平均電力$\overline{P}$や$\Delta E/N_{\mathrm{adv}}$等は従属指標として併用し,単位監査や原因切り分けに用いる.
