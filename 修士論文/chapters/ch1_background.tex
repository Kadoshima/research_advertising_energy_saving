% chapters/ch1_background.tex --- 第2章 背景
\section{背景}

\subsection{BLEアドバタイズとスキャンの基礎}
Bluetooth Low Energy(BLE)の広告は,接続を確立せずに情報をブロードキャストでき,端末側の実装が容易である\scite{bluetooth_core_spec}.一方で,広告は受信側のスキャン窓に依存して観測されるため,送信側がどれだけ制御しても「確実に届く」とは限らない.本節では,広告とスキャンの基本要素を整理する.

\subsubsection{広告イベントと3チャネル}
広告は,規格上の広告チャネル(37/38/39)を用いて送信される.送信側は,広告間隔を$T_{\mathrm{adv}}$として,概ね$T_{\mathrm{adv}}$ごとに広告イベントを発生させる.広告イベントは複数チャネルに送信されるが,周波数ホッピングや干渉により,受信側で観測される成否は確率的になる.

\subsubsection{スキャン間隔・ウィンドウとデューティ比}
受信側は,スキャン間隔$T_{\mathrm{scan}}$ごとに,スキャンウィンドウ$d_{\mathrm{scan}}$の間だけ受信機を有効化する.スキャンのデューティ比$\delta$は
\begin{equation}
  \delta=\frac{d_{\mathrm{scan}}}{T_{\mathrm{scan}}}
\end{equation}
で定義できる.$\delta$が小さいほど受信側は省電力になるが,広告イベントとの重なりが起きにくくなり,発見遅延が増える.

\subsubsection{発見遅延(TL)の概念}
イベント発生後,最初に広告が受信されるまでの遅延をTL(Time-to-first-Receive)とする.TLは,広告間隔とスキャン窓の相互作用(位相差)で分布が決まるため,平均受信率だけではQoSを十分に表現できない.本研究では,一次KPIとして省電力指標$q_{\mathrm{event}}$($\si{\micro\coulomb}/\text{event}$),TL分布の裾(TL\_p95),および期限超過率$P_{\mathrm{out}}(\tau)$を採用する.

\subsection{``非理想スキャン''という現実}
スマートフォンのスキャン挙動は,端末機種,OSバージョン,画面状態,省電力機構,および他アプリの利用状況に依存して変化する.Androidでは,アプリケーションがスキャンを要求しても,OSが内部的にスキャンの間欠化やスロットリングを行う可能性がある\scite{android_ble_scanner}.このため,本研究ではスキャンを理想的な一定窓として仮定せず,「非理想で観測できない内部状態がある」前提で送信側の制御を議論する.

\subsubsection{Valley Area(ブラインドスポット)}
広告間隔とスキャン周期が高調波関係に近い場合,広告イベントが系統的にスキャン窓の外側へ入り続け,発見遅延が極端に悪化する領域が生じ得る.このような領域は,平均的な受信率が同程度でもTLの裾を悪化させるため,$P_{\mathrm{out}}(\tau)$により議論する必要がある.

\subsubsection{Androidのスキャンモードと前提条件}
AndroidのBLEスキャンは,アプリケーションが要求するスキャンモード(例:LOW\_LATENCY)だけで挙動が一意に定まるとは限らない\scite{android_ble_scanner}.画面状態や電源管理の影響により,アプリケーションが意図したとおりに連続スキャンできない場合がある.したがって,本研究は「受信側スキャンは非理想である」前提で,送信側が取り得る適応制御の範囲を議論する.

\subsubsection{OS側のスロットリングとオポチュニスティック・スキャン}
スマートフォン側では,複数アプリのスキャン要求を統合するような挙動や,バックグラウンド制約に起因するスロットリングが生じ得る.これらはアプリケーション側から直接観測・制御できないため,評価では端末内比較と運用条件の固定(スキャンモード,画面状態等)が重要となる.

\subsubsection{端末内比較という評価姿勢}
スキャン挙動が端末・OS状態に依存する以上,異なる端末間の比較は外的要因が支配的になりやすい.本研究は,固定条件と提案条件を同一端末・同一設定で取得し,同一パイプラインで集計する端末内比較を基本とする.

\subsection{sleepをどう捉えるか}
本研究の評価は,送信側の電力と受信側のQoSを同時に扱う.このとき,sleep(間欠動作)の扱いは,受信側と送信側で意味が異なるため,整理が必要である.

\subsubsection{スキャナ側のsleep(scan dutyの縮退)}
受信側のsleepは,スキャンデューティ比$\delta$の低下として観測される.これはOSの省電力機構に起因し,アプリケーション側から直接制御できない場合が多い.スキャナ側のsleepは,TL分布の右裾を伸ばし,$P_{\mathrm{out}}(\tau)$を悪化させる方向に作用する.

\subsubsection{アドバタイザ側のsleep(平均電力の低下)}
送信側(DUT)のsleepは,CPUや無線を低消費電力状態に移行させることで平均電力を下げる.広告間隔を長くすると無線イベント回数が減るだけでなく,sleepに入れる時間が増えるため,平均電力が低下しやすい.本研究の予備計測(sleep\_eval\_scan90)では,固定\SI{100}{\milli\second}から固定\SI{500}{\milli\second}への変更で平均電力が大きく低下する一方,\SI{500}{\milli\second}以上では改善幅が小さくなる傾向が確認されている(詳細は\secref{sec:evaluation_results}).

\subsubsection{広告が支配的コストになりやすい理由}
広告イベントは,無線送信に加えてスタック処理や割り込み等を伴うため,完全なスタンバイsleepより高コストになりやすい.そのため,広告間隔の制御は「通信頻度そのものの削減」と「sleep時間の増加」という2つの経路で省電力に寄与する.

\subsection{不確実度$U$・安定度$S$・CCSの基礎}
端末内推論(HAR)は,クラス確率分布として出力を持つことが多い.この確率分布から不確実度$U$を導出すると,「いまの推論がどの程度信頼できるか」を定量化できる.一方で,確率は瞬間的に揺らぐため,時間的一貫性を表す安定度$S$も併用する設計が有効である.

\subsubsection{CCSの直感}
複合スコアCCSは,確信度$(1-U)$と安定度$S$を組み合わせた指標であり,「確信が高く,かつ安定しているほど長間隔を選びやすい」という単調な解釈を与える.この単調性は,ルールベース方策の説明容易性に寄与する.

\subsubsection{感度分析の必要性}
CCSは係数(例:$\alpha$)や窓長$W$に依存するため,分布が変わると閾値の意味が変化する.本研究では,Phase 1では説明容易性を優先し,固定の係数でルールベース評価を行う.係数感度や較正は,今後の課題として整理する(関連研究・TinyML章を参照).

\subsection{まとめ}
本節では,BLE広告とスキャンの基本要素,非理想スキャン,sleepの二面性,および不確実度指標の位置づけを整理した.次節では,これらを踏まえて関連研究を整理し,本研究の新規性と立脚点を明確化する.
