% chapters/ch1_background.tex --- 第1章 研究背景と関連研究
\chapter{研究背景と関連研究}

\section{社会背景}
ヘルスケア,見守り,労働安全などの領域では,端末が検知した状態変化を適切な遅延で外部へ通知することが求められる.しかし,常時高頻度に通信を行う設計は,端末側の電力消費を増やし,バッテリ寿命や運用コストを悪化させる.このため,必要な品質を担保しつつ,通信頻度を状況に応じて制御する仕組みが重要である.

\section{技術背景}
Bluetooth Low Energy(BLE)の広告は接続を確立せずに情報をブロードキャストでき,端末側の実装が容易である\scite{bluetooth_core_spec}.一方で,受信側(スマートフォン)のスキャンは非理想であり,OSの省電力機構や他アプリの動作によりスキャン窓が間欠化する場合がある\scite{android_ble_scanner}.この非理想性は,広告間隔とスキャン周期の相互作用により,受信遅延が長くなる領域(ブラインドスポット)が生じることを示唆する.したがって,平均的な受信率のみを目的とした制御では,最悪遅延や期限超過の観点で不十分となる.

\subsection{BLE広告間隔と到達性のトレードオフ}
広告間隔を短くすると,同じ時間内に送信される広告イベント数が増え,到達性は向上しやすい.一方で,送信回数の増加は無線イベントに伴うエネルギー消費を増大させる.このため,広告間隔は「省電力」と「到達性」のトレードオフを持つ基本パラメータである.

\subsection{スマートフォンの非理想スキャン}
受信側のスキャン挙動は,端末機種,OSバージョン,画面状態,省電力機構,および他アプリの利用状況に依存して変化する.特にバックグラウンドでは,連続スキャンが抑制され,実効デューティ比が低下する場合がある.本研究では,このような非理想性を前提として,端末内比較(同一端末・同一条件でのA/B比較)で評価を行う.

\section{評価指標(QoSと電力)}
本研究では,QoSを「遅延」と「期限超過」の観点から定義する.また,電力は平均電力に加えて,測定系のベース負荷の影響を分離できる指標も併用して解釈する.

\subsection{初回受信遅延TLと期限超過率Pout(τ)}
状態遷移(イベント)発生後,最初に広告を受信するまでの遅延をTL(Time-to-first-Receive)とする.許容遅延$\tau$に対して,$P_{\mathrm{out}}(\tau)=P(\mathrm{TL}>\tau)$を期限超過率として定義する.このとき,平均遅延だけでなく遅延分布の裾(例えば95パーセンタイル)や$P_{\mathrm{out}}(\tau)$が重要となる.

\subsection{電力指標}
平均電力は,試行(trial)期間における総エネルギーを時間で割ることで算出できる.一方で,計測系(電流計測,ロギング,LED等)の定常負荷が大きい場合,平均電力だけでは無線イベントの寄与が見えにくい.このため,本研究では「無線ONとOFFの差分」を広告イベント数で正規化した指標(例:1広告あたりの増分)も用いて議論する.

\section{既存研究における問題点}
広告間隔を固定値として最適化する方法は解析が容易であるが,環境や端末状態が時間的に変化する場合に追従できない.一方,無線状態や成功率に基づくヒューリスティックな動的制御は提案されているものの,電力と到達性を同一実験系で同時に測定し,QoS制約下の最適化として整理した事例は限定的である.また,オンライン学習(バンディット等)による適応最適化は有望であるが,学習初期のQoS違反を避けるためには,安全側のベースラインや評価指標の確立が前提となる.

\section{関連研究}
\subsection{BLE近隣発見と非理想スキャン}
BLE広告は,送信側の広告間隔と受信側のスキャン窓の相互作用により,発見遅延の分布が決まる.理想化した周期モデルでは平均遅延を解析できるが,実際のスマートフォンはOSの省電力機構によりスキャンが間欠化し,アプリケーションからはスキャン窓の実効値を観測できない場合がある.このため,本研究は「スキャンを制御する」立場ではなく,「スキャンが非理想である」前提で送信側のパラメータ(広告間隔)を制御対象とする.

\subsection{広告間隔の動的制御}
広告間隔を環境や混雑度に応じて動的に変更するアイデア自体は既存研究にも存在するが,多くはヒューリスティックなルールに留まり,QoS制約(期限超過率)と電力を同一実験系で同時に測定して整理した報告は限定的である.また,固定間隔の最適値を理論モデルから導出して適用する研究もあるが,時間変動する状況(端末状態の変化,干渉変動)に対して最適性が維持できない.

\subsection{不確実度に基づく通信制御}
機械学習の推論出力は確率分布として得られる場合が多く,不確実度を導入することで「いつ推論結果を信頼できるか」を定量化できる.この不確実度を用いて,安定期は送信頻度を下げ,遷移期は送信頻度を上げるといった適応制御は,エッジデバイスの省電力化とQoS維持の両立に有効である.本研究は,HARの不確実度を通信制御のコンテキストとして利用し,アプリ層と通信層のクロスレイヤー最適化として整理する点に特徴がある.

\subsection{バンディット学習と安全制約}
オンライン最適化(マルチアーム・バンディット等)は,環境依存性のある最適点へ逐次適応できる可能性がある\scite{lattimore2020bandit}.一方で,探索の過程でQoS制約を破るリスクがあるため,本研究では将来拡張として安全制約付きのバンディット(Safe Contextual Bandit)を想定し,その前段階としてルールベース方策と指標の確立を重視する.

\section{HARの不確実度と通信制御}
端末内推論(HAR)は,入力センサに対して確率分布として出力を持つことが多い.この確率分布から不確実度を導出すると,「いまの推論がどの程度信頼できるか」を定量化できる.本研究は,この不確実度を通信制御のコンテキストとして利用し,信頼度が高い安定期は通信を抑制し,不確実な遷移期は到達性を優先する設計を目指す.

\section{本研究の位置づけ}
本研究は,Safe Contextual Banditによるオンライン最適化を見据え,その前段階として(1)不確実度に基づくルールベース制御を定義し,(2)非理想スキャン環境下で電力と受信遅延を統合評価できる計測・指標を確立し,(3)実機で動的切替が成立する最小構成を確認することを目的とする.

\subsection{本研究の貢献}
本研究の主な貢献を以下にまとめる.
\begin{itemize}
  \item 不確実度と安定度から複合スコアを構成し,広告間隔を段階的に切り替えるルールベース方策を定義した.
  \item 送信・受信・電力計測を統合した計測基盤を整備し,平均電力とTL,$P_{\mathrm{out}}(\tau)$を同一試行で評価できる形に整理した.
  \item 実機において2値切替(\SI{100}{\milli\second}/\SI{500}{\milli\second})の最小構成を実装し,固定間隔との比較により切替が成立することを確認した.
\end{itemize}
