% chapters/ch4_rain.tex --- 第4章 評価結果
\section{評価結果}
\label{sec:evaluation_results}

\subsection{固定間隔の基準特性}
固定間隔における$q_{\mathrm{event}}$($\si{\micro\coulomb}/\text{event}$)と受信品質を基準として整理する.ただし,本節では説明の都合上,電力の従属指標として平均電力の例も併記する.特に,\SI{100}{\milli\second}から\SI{500}{\milli\second}への変更により電力が低下する一方で,受信率や遅延分布が変化するため,制約$\delta$に対する可行性が境界条件として現れる.

\subsubsection{固定間隔の平均電力(例)}
本研究では,固定間隔の平均電力を実測し,オフライン評価で用いる電力テーブルとして利用する.表\ref{tab:power_table_example}に,\SI{100}{\milli\second}から\SI{2000}{\milli\second}までの固定間隔における平均電力の一例を示す.

\begin{table}[tb]
  \centering
  \caption{固定間隔の平均電力(例)}
  \label{tab:power_table_example}
  \begin{tabular}{lll}
    \toprule
    広告間隔 [ms] & 平均電力 [mW] & 備考 \\
    \midrule
    100  & 198.56 & n=10(sleep\_on) \\
    500  & 180.80 & n=9(sleep\_on) \\
    1000 & 178.62 & n=9(sleep\_on) \\
    2000 & 177.47 & n=9(sleep\_on) \\
    \bottomrule
  \end{tabular}
\end{table}

\subsubsection{主効果の観察}
表\ref{tab:power_table_example}の例では,\SI{100}{\milli\second}から\SI{500}{\milli\second}への変更で平均電力が大きく低下する.一方で,\SI{500}{\milli\second}以上では改善幅が小さくなる.この傾向は,動的制御において短間隔の滞在時間を抑制する設計が有効であることを示唆する.

\subsection{動的切替(2値制御)の実機確認}
動的制御の最小構成として,\SI{100}{\milli\second}と\SI{500}{\milli\second}の切替を実装し,固定\SI{100}{\milli\second}および固定\SI{500}{\milli\second}と比較する.平均電力が両固定条件の中間に位置し,受信率も中間的になることを確認することで,切替が成立していることを検証する.

切替成立の確認としては,平均電力が両固定条件の中間に位置し,かつログ上でinterval切替が発生することを確認すれば十分である.以降では,TLと$P_{\mathrm{out}}(\tau)$を用いて,QoS制約下での省電力効果を定量評価する.

\subsection{実測のQoS指標(TLとPout)}
本研究では,受信品質をPDRだけでなく,遅延分布と期限超過率$P_{\mathrm{out}}(\tau)$で評価する.特に,非理想スキャン環境では平均受信率が同程度でも,遅延の裾が悪化する場合があるため,$P_{\mathrm{out}}(\tau)$が重要となる(\secref{sec:metrics_detail}).

\subsubsection{実測例(D2b,scan90)}
表\ref{tab:d2_summary}に,D2b(scan90)における実測例を示す.本表は,S1/S4の2条件について,固定\SI{100}{\milli\second},固定\SI{500}{\milli\second},および方策(2値切替)の比較をまとめたものである.ここでは run B(n=3)と追加取得 B/02(n=3)を統合し,各条件n=6として集計した.

\begin{table}[tb]
  \centering
  \caption{D2b(scan90)の実測例(mean$\pm$std, 各n=6)}
  \label{tab:d2_summary}
  {\small
  \setlength{\tabcolsep}{4pt}
  \begin{tabular}{lrrrrr}
    \toprule
    条件 & Pout(1s) & TL\_mean[s] & PDR\_u & P[mW] & share100 \\
    \midrule
    S1-100   & 0.075$\pm$0.027 & 3.76$\pm$1.46 & 0.789$\pm$0.016 & 204.1$\pm$1.4 & 1.000$\pm$0.000 \\
    S1-500   & 0.142$\pm$0.049 & 5.29$\pm$0.02 & 0.816$\pm$0.018 & 184.7$\pm$1.5 & 0.000$\pm$0.000 \\
    S1-policy & 0.125$\pm$0.027 & 5.24$\pm$0.05 & 0.803$\pm$0.022 & 191.5$\pm$1.9 & 0.331$\pm$0.004 \\
    \midrule
    S4-100   & 0.053$\pm$0.010 & 1.25$\pm$0.01 & 0.792$\pm$0.009 & 204.4$\pm$2.1 & 0.998$\pm$0.002 \\
    S4-500   & 0.146$\pm$0.031 & 2.48$\pm$1.11 & 0.817$\pm$0.020 & 184.5$\pm$1.5 & 0.000$\pm$0.000 \\
    S4-policy & 0.069$\pm$0.029 & 1.58$\pm$0.50 & 0.793$\pm$0.021 & 196.6$\pm$1.6 & 0.595$\pm$0.008 \\
    \bottomrule
  \end{tabular}
  }
\end{table}

\begin{figure}[tb]
  \centering
  \includegraphics[width=0.90\linewidth]{../uccs_d2_scan90/plots/d2b_B_n6_power_vs_pout}
  \caption{D2b(run B + B/02, n=6)における平均電力と$P_{\mathrm{out}}(1\mathrm{s})$の関係(share100を注釈)}
  \label{fig:d2b_power_vs_pout}
\end{figure}

\subsubsection{解釈}
表\ref{tab:d2_summary}および図\ref{fig:d2b_power_vs_pout}より,方策(2値切替)はFixed100より低電力であり,Fixed500より低い期限超過率(QoS改善)を示す運用点になり得る.さらに,遷移が多い側(S4)ほどshare100が増加しており,「必要時だけ\SI{100}{\milli\second}に寄せ,それ以外は\SI{500}{\milli\second}に寄せる」挙動が定量で確認できる.

また,平均電力はFixed100/Fixed500の滞在比率(share100)による線形混合で概ね説明できるため,方策の省電力効果はinterval滞在比率に支配されることが示唆される.これにより,「sleep差分が効いた/効かなかった」といった実装依存の議論から切り離し,制御則(どの状態で短間隔へ戻すか)の議論へ接続しやすくなる.

\subsubsection{条件悪化時の頑健性(D3,scan70)}
\label{sec:d3_scan70}
表\ref{tab:d3_summary}および図\ref{fig:d3_d4_power_vs_pout}(a)に,scan dutyを低下させたD3(scan70, S4, 各n=3)の結果を示す.scan70ではFixed500の期限超過率が0.285まで悪化する一方で,方策は0.089まで改善し,かつ平均電力はFixed100(209.9\,mW)より低い(202.1\,mW).

なお,scan70では受信欠落によりRXタグ由来のshare100が過小評価されやすいため,D3では平均電力の線形混合から推定したshare100\_mixを併記した.また,SDコピーでmtimeが信頼できないため,TXSDログはadv\_count(tick\_count)でクラスタリングしてRXと対応付けた.

\begin{table}[tb]
  \centering
  \caption{D3(scan70, S4)の実測例(mean$\pm$std, 各n=3)}
  \label{tab:d3_summary}
  {\small
  \setlength{\tabcolsep}{4pt}
  \begin{tabular}{lrrrr}
    \toprule
    条件 & Pout(1s) & TL\_mean[s] & P[mW] & share100\_mix \\
    \midrule
    S4-100   & 0.065$\pm$0.028 & 1.34$\pm$0.02 & 209.9$\pm$0.5 & 1.000 \\
    S4-500   & 0.285$\pm$0.061 & 2.93$\pm$1.07 & 189.5$\pm$0.5 & 0.000 \\
    S4-policy & 0.089$\pm$0.037 & 1.40$\pm$0.07 & 202.1$\pm$0.1 & 0.618 \\
    \bottomrule
  \end{tabular}
  }
\end{table}

\subsubsection{U-shuffleアブレーション(D4,scan90)}
\label{sec:d4_ablation}
D2bの主結果に対して,「不確実度$U$は本当に効いているか(単なる閾値遊びではないか)」という疑問に答えるため,D4としてアブレーション実験を行った.ここでは制御器の構造と閾値は固定し,不確実度系列$U$の時間整合だけを破壊する(shuffleする)ことで,制御がどのように崩れるかを観察する.

表\ref{tab:d4_summary}および図\ref{fig:d3_d4_power_vs_pout}(b)に,S4(遷移が多い条件)における結果(各n=3)を示す.方策(U+CCS)はFixed100より低電力であり,Fixed500より低い期限超過率を示す運用点になり得る.一方で,$U$をshuffleするとshare100が0.593から0.943へ増加し,adv\_countも1227から1715へ増加してFixed100に近い挙動へ崩れる.その結果,平均電力も200.5\,mWから208.1\,mWへ増加し,省電力効果が失われる.

\begin{table}[tb]
  \centering
  \caption{D4(scan90, S4)のアブレーション結果(mean$\pm$std, 各n=3)}
  \label{tab:d4_summary}
  {\small
  \setlength{\tabcolsep}{4pt}
  \begin{tabular}{lrrrrr}
    \toprule
    条件 & Pout(1s) & TL\_mean[s] & P[mW] & adv\_count & share100 \\
    \midrule
    Fixed100 & 0.049$\pm$0.000 & 1.24$\pm$0.01 & 208.2$\pm$1.3 & 1796$\pm$0 & 1.000$\pm$0.000 \\
    Fixed500 & 0.130$\pm$0.014 & 1.59$\pm$0.06 & 187.9$\pm$0.8 & 359$\pm$0 & 0.000$\pm$0.000 \\
    Policy(U+CCS) & 0.098$\pm$0.024 & 1.28$\pm$0.01 & 200.5$\pm$0.8 & 1227$\pm$0 & 0.593$\pm$0.009 \\
    Ablation(U-shuf) & 0.049$\pm$0.000 & 1.23$\pm$0.01 & 208.1$\pm$0.3 & 1715$\pm$0 & 0.943$\pm$0.000 \\
    \bottomrule
  \end{tabular}
  }
\end{table}

\begin{figure}[tb]
  \centering
  \begin{minipage}{0.48\linewidth}
    \centering
    \includegraphics[width=\linewidth]{../uccs_d3_scan70/plots/d3_01_power_vs_pout}
    {\small (a) D3: scan70(S4, n=3)\par}
  \end{minipage}\hfill
  \begin{minipage}{0.48\linewidth}
    \centering
    \includegraphics[width=\linewidth]{../uccs_d4_scan90/plots/d4_01_power_vs_pout}
    {\small (b) D4: U-shuffle(scan90, S4, n=3)\par}
  \end{minipage}
  \caption{条件悪化(D3)およびU-shuffleアブレーション(D4)における平均電力と$P_{\mathrm{out}}(1\mathrm{s})$の関係(share100を注釈)}
  \label{fig:d3_d4_power_vs_pout}
\end{figure}

\subsubsection{解釈(D4)}
D4の結果は,制御器の構造と閾値を固定したまま入力$U$の時間整合だけを破壊すると,制御が短間隔(\SI{100}{\milli\second})に張り付く方向へ崩れることを示している.したがって,$U$(およびCCS)の時間的な整合は「必要時だけ短間隔へ寄せる」ための鍵であり,省電力とQoSの両立に寄与することが示唆される.

\subsection{オフライン評価との整合}
オフライン評価で予測される運用点と,実機で得られる平均電力・受信品質の差分を比較し,推定モデルの妥当性と限界を整理する.

\subsubsection{予測と実測の差分}
オフライン評価は,固定間隔の電力テーブルと受信品質推定を合成するため,実環境の非理想性を完全には表現しない.したがって,予測値は「候補探索と説明のための近似」として用い,実測との差分を評価しながらモデルの限界を記述することが重要である.

\subsubsection{δ帯プロット(参考)}
δ帯プロットは,\secref{sec:offline_eval}の図\ref{fig:letter_delta_band}に示した通り,固定間隔点と候補方策点の位置関係を俯瞰し,実機評価で探索すべき領域を絞り込むための近似として有効である.

\subsection{まとめ}
本節の結果を,次章の考察に接続するために要点を整理する.
\begin{itemize}
  \item D2b(scan90, n=6)では,方策(2値切替)がFixed100より低電力で,Fixed500より低い期限超過率を示す運用点になり得ることを確認した.
  \item D3(scan70, S4, n=3)では,scan duty低下でFixed500が崩れる条件でも,方策が期限超過率を抑えつつFixed100より低電力を維持できることを確認した.
  \item D4(U-shuffleアブレーション, S4, n=3)では,$U$の時間整合を破壊すると短間隔へ張り付く方向へ崩れ,省電力効果が失われることを確認した.
  \item オフライン評価(δ帯)は,固定間隔点と候補方策点の位置関係を可視化し,実機評価で探索すべき領域を絞り込むための近似として有効である.
\end{itemize}
