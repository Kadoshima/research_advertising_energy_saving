% chapters/ch4_rain.tex --- 第4章 評価結果
\section{評価結果}
\label{sec:evaluation_results}

\subsection{固定間隔の基準特性}
固定間隔における$q_{\mathrm{event}}$($\si{\micro\coulomb}/\text{event}$)と受信品質を基準として整理する.ただし,本節では説明の都合上,電力の従属指標として平均電力の例も併記する.特に,\SI{100}{\milli\second}から\SI{500}{\milli\second}への変更により電力が低下する一方で,受信率や遅延分布が変化するため,制約$\delta$に対する可行性が境界条件として現れる.

\subsubsection{固定間隔の平均電力(例)}
本研究では,固定間隔の平均電力を実測し,オフライン評価で用いる電力テーブルとして利用する.表\ref{tab:power_table_example}に,\SI{100}{\milli\second}から\SI{2000}{\milli\second}までの固定間隔における平均電力の一例を示す.

\begin{table}[tb]
  \centering
  \caption{固定間隔の平均電力(例)}
  \label{tab:power_table_example}
  \begin{tabular}{lll}
    \toprule
    広告間隔 [ms] & 平均電力 [mW] & 備考 \\
    \midrule
    100  & 198.56 & n=10(sleep\_on) \\
    500  & 180.80 & n=9(sleep\_on) \\
    1000 & 178.62 & n=9(sleep\_on) \\
    2000 & 177.47 & n=9(sleep\_on) \\
    \bottomrule
  \end{tabular}
\end{table}

\subsubsection{主効果の観察}
表\ref{tab:power_table_example}の例では,\SI{100}{\milli\second}から\SI{500}{\milli\second}への変更で平均電力が大きく低下する.一方で,\SI{500}{\milli\second}以上では改善幅が小さくなる.この傾向は,動的制御において短間隔の滞在時間を抑制する設計が有効であることを示唆する.

\subsection{動的切替(2値制御)の実機確認}
動的制御の最小構成として,\SI{100}{\milli\second}と\SI{500}{\milli\second}の切替を実装し,固定\SI{100}{\milli\second}および固定\SI{500}{\milli\second}と比較する.平均電力が両固定条件の中間に位置し,受信率も中間的になることを確認することで,切替が成立していることを検証する.

本研究の実測例として,固定\SI{100}{\milli\second}(n=3),固定\SI{500}{\milli\second}(n=3),動的(n=3)における平均電力と受信率のサマリを表\ref{tab:d1_summary}に示す.数値は実装・環境に依存するため,本文執筆時点のログに基づき更新する.

\begin{table}[tb]
  \centering
  \caption{動的切替(2値制御)の実測例(サマリ)}
  \label{tab:d1_summary}
  \begin{tabular}{llll}
    \toprule
    条件 & 平均電力 [mW] & 受信率の目安 [Hz] & 備考 \\
    \midrule
    固定100 & 205.10 $\pm$ 2.79 & 7.94 $\pm$ 0.21 & scan90 \\
    固定500 & 185.50 $\pm$ 0.98 & 1.63 $\pm$ 0.09 & scan90 \\
    動的(100/500) & 192.17 $\pm$ 0.55 & 3.71 $\pm$ 0.11 & 切替動作確認 \\
    \bottomrule
  \end{tabular}
\end{table}

\subsection{実測のQoS指標(TLとPout)}
本研究では,受信品質をPDRだけでなく,遅延分布と期限超過率$P_{\mathrm{out}}(\tau)$で評価する.特に,非理想スキャン環境では平均受信率が同程度でも,遅延の裾が悪化する場合があるため,$P_{\mathrm{out}}(\tau)$が重要となる(\secref{sec:metrics_detail}).

\subsubsection{実測例(D2b,scan90)}
表\ref{tab:d2_summary}に,D2b(scan90)における実測例を示す.本表は,S1/S4の2条件について,固定\SI{100}{\milli\second},固定\SI{500}{\milli\second},および方策(2値切替)の比較をまとめたものである.ここでは run B(n=3)と追加取得 B/02(n=3)を統合し,各条件n=6として集計した.

\begin{table}[tb]
  \centering
  \caption{D2b(scan90)の実測例(mean$\pm$std, 各n=6)}
  \label{tab:d2_summary}
  {\small
  \setlength{\tabcolsep}{4pt}
  \begin{tabular}{lrrrrr}
    \toprule
    条件 & Pout(1s) & TL\_mean[s] & PDR\_u & P[mW] & share100 \\
    \midrule
    S1-100   & 0.075$\pm$0.027 & 3.76$\pm$1.46 & 0.789$\pm$0.016 & 204.1$\pm$1.4 & 1.000$\pm$0.000 \\
    S1-500   & 0.142$\pm$0.049 & 5.29$\pm$0.02 & 0.816$\pm$0.018 & 184.7$\pm$1.5 & 0.000$\pm$0.000 \\
    S1-policy & 0.125$\pm$0.027 & 5.24$\pm$0.05 & 0.803$\pm$0.022 & 191.5$\pm$1.9 & 0.331$\pm$0.004 \\
    \midrule
    S4-100   & 0.053$\pm$0.010 & 1.25$\pm$0.01 & 0.792$\pm$0.009 & 204.4$\pm$2.1 & 0.998$\pm$0.002 \\
    S4-500   & 0.146$\pm$0.031 & 2.48$\pm$1.11 & 0.817$\pm$0.020 & 184.5$\pm$1.5 & 0.000$\pm$0.000 \\
    S4-policy & 0.069$\pm$0.029 & 1.58$\pm$0.50 & 0.793$\pm$0.021 & 196.6$\pm$1.6 & 0.595$\pm$0.008 \\
    \bottomrule
  \end{tabular}
  }
\end{table}

\begin{figure}[tb]
  \centering
  \includegraphics[width=0.90\linewidth]{../uccs_d2_scan90/plots/d2b_B_n6_power_vs_pout}
  \caption{D2b(run B + B/02, n=6)における平均電力と$P_{\mathrm{out}}(1\mathrm{s})$の関係(share100を注釈)}
  \label{fig:d2b_power_vs_pout}
\end{figure}

\subsubsection{解釈}
表\ref{tab:d2_summary}および図\ref{fig:d2b_power_vs_pout}より,方策(2値切替)はFixed100より低電力であり,Fixed500より低い期限超過率(QoS改善)を示す運用点になり得る.さらに,遷移が多い側(S4)ほどshare100が増加しており,「必要時だけ\SI{100}{\milli\second}に寄せ,それ以外は\SI{500}{\milli\second}に寄せる」挙動が定量で確認できる.

また,平均電力はFixed100/Fixed500の滞在比率(share100)による線形混合で概ね説明できるため,方策の省電力効果はinterval滞在比率に支配されることが示唆される.これにより,「sleep差分が効いた/効かなかった」といった実装依存の議論から切り離し,制御則(どの状態で短間隔へ戻すか)の議論へ接続しやすくなる.

\subsection{オフライン評価との整合}
オフライン評価で予測される運用点と,実機で得られる平均電力・受信品質の差分を比較し,推定モデルの妥当性と限界を整理する.

\subsubsection{予測と実測の差分}
オフライン評価は,固定間隔の電力テーブルと受信品質推定を合成するため,実環境の非理想性を完全には表現しない.したがって,予測値は「候補探索と説明のための近似」として用い,実測との差分を評価しながらモデルの限界を記述することが重要である.

\subsubsection{δ帯プロット(参考)}
図\ref{fig:delta_band_again}に,オフライン評価の代表例としてδ帯プロットを示す.固定間隔点と候補方策点を同一平面で可視化することで,制約境界付近における省電力運用点の存在を示す.

\begin{figure}[tb]
  \centering
  \includegraphics[width=0.95\linewidth]{../results/mhealth_policy_eval/letter_v4_scan90_v5_delta_tight_sleep_on_n9_10_actions_100_500/fig_delta_band.png}
  \caption{δ帯(例)におけるPout(1s)と平均電力のトレードオフ(オフライン評価)}
  \label{fig:delta_band_again}
\end{figure}

\subsection{まとめ}
本節の結果を,次章の考察に接続するために要点を整理する.
