% chapters/appx_e_runbook.tex --- 付録E 実験Runbook(チェックリスト)
\section{実験Runbook(チェックリスト)}
\label{sec:runbook}

本節では,実験の再現性を担保するための最小チェックリストを示す.運用中に発見された落とし穴(条件混在,欠損,単位不整合など)を避けることを目的とする.

\subsection{実験前チェック}
\begin{itemize}
  \item 役割の確認:TX/RX/TXSDでスケッチが正しい(接頭辞,設定値).
  \item 配線の確認:GND共通,SYNC入力,必要ならTICK入力(GPIO番号).
  \item 受信端末の状態:スキャンモード,画面状態,省電力設定を固定し,端末内比較の前提を守る.
  \item SD空き容量:試行本数に対して十分な空きがある.
  \item ログ命名:\texttt{\detokenize{trial_XXX.csv}}の連番と,条件IDの対応が追跡できる.
\end{itemize}

\subsection{試行(trial)実行}
\begin{enumerate}
  \item SYNCで試行開始が一致していることを目視(LED等)で確認する.
  \item 試行中は条件(広告間隔,スキャン設定)を変更しない.
  \item 試行終了後,TXSDログ末尾のsummary(ms\_total, adv\_count, E\_total等)を確認する.
  \item 明らかな異常(rate低下,欠損,桁ズレ)があれば,その場で再試行し,除外理由を作業ログに残す.
\end{enumerate}

\subsection{データ整理}
\begin{itemize}
  \item 生データの退避:\texttt{\detokenize{data/}}へコピーし,取得条件と紐付ける.
  \item 解析対象の選別:含めるtrialと除外trialを明示し,理由(欠損,条件混在等)を記録する.
  \item ハッシュ記録:重要データセットはSHA256を記録し,後から同一入力で再現できるようにする.
  \item 作業ログ追記:生成物のパスと実行コマンドを\texttt{\detokenize{logs/worklog_YYYY-MM-DD_*.txt}}へ追記する.
\end{itemize}

\subsection{解析と図表生成}
\begin{itemize}
  \item trial集計:ログを読み込み,欠損と単位を監査し,指標(PDR, TL, Pout, 平均電力)を算出する.
  \item 集約:条件ごとに平均と分散を計算し,図表(固定点,δ帯,Pareto等)を生成する.
  \item 再現確認:同一入力・同一コマンドで同じ図表が出ることを確認する(乱数がある場合はseed固定).
  \item 図の固定:提出用には\texttt{\detokenize{修士論文/figures/}}へコピーし,相対パス参照を解消する.
\end{itemize}

\subsection{既知の注意点}
\begin{itemize}
  \item 文字列中の\texttt{\detokenize{_}}(アンダースコア)は,LaTeX本文では\texttt{\detokenize{\texttt{\detokenize{...}}}}等で扱う.
  \item TL/Poutは開始位相ずれで歪むため,定数オフセット補正を適用した定義(v5)を基準にする(\secref{sec:stress_fixed_metrics}).
  \item 条件混在(LED/SYNCの扱いの違い等)は定常オフセットとなるため,比較は同一コード系列に限定する(\secref{sec:measurement_system}).
\end{itemize}
