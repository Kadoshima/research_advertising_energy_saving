% chapters/ch10_implementation.tex --- 第10章 実装(ファームウェア・解析パイプライン)
\chapter{実装:ファームウェアと解析パイプライン}

\section{目的}
本研究では,方策の比較を可能にするため,計測系と解析パイプラインを共通化した.本章では,実装の要点(ログ設計,同期,解析フロー)を整理する.

\section{ファームウェア構成}
ファームウェアは役割ごとに送信(TX),受信(RX),電力ロガ(TXSD)に分割する.各スケッチは,役割がファイル名から判別できるよう命名し,試行境界の同期とログ出力形式を統一する.

\subsection{命名規約}
実験の再現性を担保するため,スケッチ名は役割別に接頭辞を付与する.
\begin{itemize}
  \item RX\_*:受信ロガ(RX)
  \item TX\_*:送信・被測定対象(TX)
  \item TXSD\_*:送信側の電力ロガ(TXSD)
\end{itemize}
この規約により,ファイル名だけで役割が判別でき,Runbookや作業ログから参照しやすくなる.

\section{ログ形式}
本研究の解析は,TXSDログ(電力)とRXログ(受信)を中心に行う.特に,広告回数(adv\_count)や試行時間(ms\_total)の扱いは,指標の分母・分子に直結するため,ログ末尾のsummaryを正として扱う.

\subsection{解析で重要な列}
TXSDログでは,時刻(ms)と計測値(mV,$\mu$A)が一次情報であり,平均電力や総エネルギーはここから再計算できる.RXログでは,受信時刻とseqがTL/PoutおよびPDR\_uniqueの計算に必須である.したがって,ログ設計では「後から復元できる列」を優先し,表示用の派生列は監査可能な形で保持する.

\section{解析パイプラインの流れ}
解析は,概ね以下の段階で行う.
\begin{enumerate}
  \item trialごとのログを読み込み,欠損や単位の整合をチェックする.
  \item TXSDから平均電力・総エネルギー等を算出する.
  \item RXからPDR,TL,Pout(τ)を算出する(必要に応じて時間同期を行う).
  \item 条件ごとに集約し,図表を生成する.
\end{enumerate}

\subsection{代表的なスクリプト}
解析は,試行ごとの集計,条件ごとの集約,図表生成に分割することで,再実行と検証を容易にする.例えばストレス固定(第6章)では,時間同期を含む集計と,集約・図表化を分離して実行できるように整備した.

\section{再現性のための運用}
解析対象の選別(除外理由),生成コマンド,入力データの出典・ハッシュ等を記録し,再現可能な形で結果を残す.作業ログとインデックスを併用して,後から辿れる状態を維持する.

\section{まとめ}
本章では,ファームウェアと解析パイプラインの要点を整理した.これにより,固定間隔と動的制御を同一基盤で比較できる.
