% chapters/ch2_related_work.tex --- 第3章 関連研究
\section{関連研究}

\subsection{BLE広告・近隣発見の最適化}
BLE広告における主要設計パラメータは,広告間隔,送信チャネル数,送信電力等である.これらは到達性(受信率・遅延)と電力のトレードオフに直結するため,固定値の最適化や,状況に応じた動的制御が検討されてきた.

\subsubsection{MABでintervalを選ぶ考え方}
広告間隔の動的制御は,混雑度推定や直近の受信状況に基づくルールとして実装されることが多い.一方で,未知・非定常な環境では,有限個の行動集合からオンラインで運用点を学習する枠組み(マルチアーム・バンディット)も候補となる\scite{lattimore2020bandit}.本研究はPhase 2での拡張可能性を残しつつ,Phase 1では学習以前の前提(指標・計測)を確立する立場を採る.

\subsubsection{3チャネル送信と電力}
BLE広告は広告チャネル(37/38/39)で送信されるため,チャネル数や送信回数は到達性と電力に影響する.本研究はPhase 1ではチャネルマスク等を固定し,広告間隔の効果と不確実度駆動制御の枠組みに焦点を当てる.

\subsubsection{学習しない設計(多重化等)}
学習を用いないアプローチとしては,複数の間隔や送信パターンを組み合わせ,Valley Areaの影響を平均化する設計も考えられる.ただし,受信側スキャンが非理想である場合,無線側のパラメータだけで性能が決まらないため,アプリケーション層の状態(遷移期/安定期)を併用した説明が必要となる.

\subsection{遅延制約・Outage解析}
近隣発見では,平均遅延だけでなく,期限超過率$P_{\mathrm{out}}(\tau)$や遅延分布の裾(例:TL\_p95)が重要である.スキャン窓のデューティ比が低い場合,平均受信率が一定でも遅延分布が右に伸び,$P_{\mathrm{out}}(\tau)$が支配的な評価軸になる.

理想化した周期モデルでは,広告イベントとスキャン窓の相互作用から遅延分布を解析できるが,実際のスマートフォンはOSの省電力機構によりスキャンが間欠化し,観測できない内部状態が存在する.したがって,本研究では解析モデルによる厳密導出ではなく,実測ログに基づき$P_{\mathrm{out}}(\tau)$と電力のトレードオフを描き,方策比較を端末内比較として解釈する.

\subsubsection{scan duty・sleepとoutageの関係}
受信側のsleepは,スキャンデューティ比の縮退として現れ,TL分布の裾を悪化させる方向に作用する.一方で送信側のsleepは,広告間隔の延伸と組み合わせることで平均電力を下げる.この二面性を踏まえ,sleepは背景・装置・考察に分散して扱う必要がある.

\subsection{不確実度駆動の通信制御(クロスレイヤー)}
推論モデルが出力する確率分布は,不確実度として定量化できる.不確実度に基づいて「重要なときだけ通信する」「確信が低いときだけサンプリングや送信を増やす」といった制御は,エッジAIにおける省電力化の代表的な戦略である.

ただし,不確実度の定義や較正が不適切であると,確信が過大評価されて長間隔に偏り,QoS違反が増える危険がある.本研究では,確率較正の必要性を踏まえつつ(将来課題),まずは確率分布から導出した$U$と時間的一貫性$S$を組み合わせ,説明可能な複合スコアCCSとして通信制御へ接続する.

\subsection{TinyML環境での制約付き学習・実装論}
MCU上で学習・制御を実装する場合,RAMや演算資源の制約により,実装可能なアルゴリズムが制限される.現実的には,学習器そのものは単純な最適化に徹し,その外側にSafety Filter(Action Masking等)を置いてハード制約を担保する構造が採られることが多い.

本研究の行動集合(広告間隔)は小さいため,Phase 2での拡張としては,低次元のコンテキストに対する軽量なContextual Banditが候補となる.一方,Phase 1の段階では,実験計測と指標定義がボトルネックになるため,まずはルールベース方策で再現性の高いデータを取得することが重要である.

\subsection{オンライン最適化(バンディット)と安全制約}
医療・見守り等の用途ではQoS違反が許容されにくく,探索の過程で制約を破るリスクが課題となる.このため,本研究は,Phase 1として安全側のルールベース方策と評価指標を確立した上で,将来課題として安全制約付きのContextual Bandit(Safe Contextual Bandit)へ接続する方針を採る.

\subsection{本研究の位置づけ}
表\ref{tab:related_work_positioning}に,本研究と関連アプローチの整理を示す.本研究は,「不確実度を用いた動的制御」と「電力+QoSの同時実測」を同一の評価系で統合し,Phase 2の安全制約付きオンライン最適化へ接続できる形に整理する点に特徴がある.

\begin{table}[tb]
  \centering
  \caption{関連アプローチと本研究の位置づけ(概念整理)}
  \label{tab:related_work_positioning}
  \begin{tabular}{p{0.22\linewidth}p{0.22\linewidth}p{0.22\linewidth}p{0.22\linewidth}}
    \toprule
    区分 & 主な入力 & 制御の形式 & 評価の主軸 \\
    \midrule
    固定設計 & 規格・経験値 & 固定$a$ & 平均電流,PDR \\
    ルールベース & 混雑度・直近受信 & 閾値で切替 & 平均電力,遅延 \\
    バンディット & 観測(報酬) & 探索+活用 & 報酬最大化 \\
    本研究 & $U,S,\mathrm{CCS}$ & ルール(Phase 1) & $\overline{P}$,$P_{\mathrm{out}}(\tau)$,TL\_p95 \\
    \bottomrule
  \end{tabular}
\end{table}
