% chapters/appx_c_results.tex --- 付録C 追加結果(図表)
\chapter{追加結果}

\section{オフライン評価の比較表}
図\ref{fig:policy_table}に,固定とルールベース方策の比較表(例)を示す.

\begin{figure}[tb]
  \centering
  \includegraphics[width=0.95\linewidth]{../results/mhealth_policy_eval/policy_table.png}
  \caption{方策の比較表(例)}
  \label{fig:policy_table}
\end{figure}

\section{Paretoプロット(追加)}
図\ref{fig:pareto_plots_appx}に,オフライン評価のParetoプロットの例を示す.本文ではδ帯プロットを主張図として用いたが,探索空間全体の分布を併せて示すことで,候補集合の位置づけが明確になる.

\begin{figure}[tb]
  \centering
  \includegraphics[width=0.95\linewidth]{../results/mhealth_policy_eval/pareto_front_v8_power_table_scan90_v5_sleep_on_n9_10_actions_100_500/pareto_plots.png}
  \caption{Paretoプロット(例)}
  \label{fig:pareto_plots_appx}
\end{figure}

\section{ストレス固定:実測と簡易モデル}
図\ref{fig:stress_fixed_real_vs_sim_appx}に,ストレス固定における実測と簡易モデルの比較図を再掲する(本文第6章).

\begin{figure}[tb]
  \centering
  \includegraphics[width=0.95\linewidth]{../results/stress_fixed/compare/stress_causal_real_vs_sim.png}
  \caption{ストレス固定における実測と簡易モデルの比較(例)}
  \label{fig:stress_fixed_real_vs_sim_appx}
\end{figure}

\section{備考}
本付録の図表は,本文の主張に直接必要でないが,読者が条件や比較対象を追跡できるように掲載する.
