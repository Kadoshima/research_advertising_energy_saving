% chapters/appx_g_troubleshooting.tex --- 付録G トラブルシューティング集
\section{トラブルシューティング集}
\label{sec:troubleshooting}

本節では,実験運用で経験した代表的なトラブルを整理し,原因の切り分けと対策をまとめる.目的は,欠損や時間軸不整合によって結論が崩壊することを避け,再現性を高めることである.

\subsection{症状からの切り分け}
\begin{longtable}{p{0.28\linewidth}p{0.32\linewidth}p{0.34\linewidth}}
  \caption{症状からの切り分け(例)}\label{tab:troubleshooting_by_symptom}\\
  \toprule
  症状 & 疑う箇所 & まず確認すること \\
  \midrule
  \endfirsthead
  \toprule
  症状 & 疑う箇所 & まず確認すること \\
  \midrule
  \endhead
  OFFがONより大きい($\Delta E<0$) & 欠損,単位換算,条件混在 & 生ログ再積分とsummaryの一致,コード系列の統一 \\
  TL/Poutが不自然に小さい & RXとtruthの時間軸ずれ & 定数オフセット補正(v5)を適用したか \\
  PDRが1を超える & 重複受信の混入 & seqユニーク化(PDR\_unique)を優先 \\
  adv\_countが期待値と乖離 & TICK未接続,推定の誤差 & TICK物理カウントの有無,推定式の前提 \\
  SDログが途中で途切れる & SD I/O,電源,ファイルopen & 末尾summaryの有無,open/initエラー \\
  数値列に文字が混入する & UART品質,ボーレート過大 & ボーレート低下,厳密パース,配線見直し \\
  \bottomrule
\end{longtable}

\subsection{代表的トラブル}
\subsubsection{広告OFFの方が消費が大きい($\Delta E<0$)}
広告OFFが広告ONより大きい結果は,物理的に矛盾して見えるため,最優先で切り分ける必要がある.本研究では,無線スタックの影響を議論する前に,計測・解析起因の破綻を疑う.
\begin{itemize}
  \item 主要因候補:I/Oボトルネックによる欠損,単位換算ミス,条件混在(コード系列の違い).
  \item 対策:生ログ(mV/µA)から再積分し,summaryと一致することを監査する.比較は同一コード系列で統一し,sleepの有無も条件として明示する.
  \item 判断:監査を通過した上で$\Delta E<0$が再現する場合,電源ドメインや無線状態遷移など物理要因を再検討する.
\end{itemize}

\subsubsection{TL/Poutの過小評価(開始位相ずれ)}
RXログの開始がtruthより遅れる等の理由で,RXの時間軸とtruthの時間軸が一致しない場合がある.このずれを補正しないと,TLや$P_{\mathrm{out}}(\tau)$が過小評価される.本研究ではv5として,seqから定数オフセットを推定して補正する手順を採用する(\secref{sec:stress_fixed_metrics}).

\subsubsection{UARTデータ化け(数値列への文字混入)}
UARTログが壊れると,欠損や桁ズレが連鎖し,積分と分母(試行時間・広告回数)が不整合になる.
\begin{itemize}
  \item 兆候:パーサの例外,行数不足,msが単調増加しない,0埋め行の増加.
  \item 対策:ボーレートを下げる,出力列を最小化する,パース失敗は試行ごと除外し理由を記録する.
  \item 再発防止:末尾summaryの生成を必須とし,欠損率が閾値を超えた試行は自動で失格にする.
\end{itemize}

\subsubsection{SYNC検出失敗(短パルス)}
SYNCが短パルスとして出力されると,受信側で取りこぼす可能性がある.境界が崩れると,TL/Poutの定義(イベント時刻)と電力積分区間が一致しなくなる.
\begin{itemize}
  \item 対策:SYNCをHIGH保持する,受信側でラッチする,開始と終了の両方を同期する.
  \item 監査:TXSD summaryのms\_totalが想定長と一致すること,RXの試行数が一致することを確認する.
\end{itemize}

\subsubsection{SD I/O失敗(open/init)}
SDカード初期化やファイルopenに失敗した状態で試行を継続すると,欠損が増え,評価が不能になる.
\begin{itemize}
  \item 対策:SPIクロック調整,塊書き込み,open/init失敗時の即時停止と再試行.
  \item 監査:各trialファイルの末尾summaryの存在と整合を確認する.
\end{itemize}

\subsection{解析前の最終監査}
解析を自動化しても,入力が壊れていると結論が壊れる.したがって,解析前に以下を最低限チェックする.
\begin{itemize}
  \item 末尾summary(ms\_total, adv\_count, E\_total等)が存在し,桁が妥当である.
  \item 生ログの再積分がsummaryと整合し,単位換算が一貫している.
  \item RXの試行数・条件IDがTXSDと一致し,条件混在がない.
  \item 重要データはSHA256を記録し,同一入力で再現できる.
\end{itemize}
