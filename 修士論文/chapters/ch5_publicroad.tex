% chapters/ch5_publicroad.tex --- 第5章 考察
\chapter{考察}
\clearpage

\section{非理想スキャン環境における解釈}
受信側スキャンの非理想性は,平均的な受信率だけでなく,遅延の裾や期限超過に影響する.端末・OS状態に依存する要素を前提として,評価は端末内のA/B比較として解釈する必要がある.

\subsection{評価の前提}
本研究では,端末内比較の前提を守るため,固定条件と提案条件を同一実験系で取得し,同じ処理パイプラインで集計する.また,ログ欠損や同期ずれが評価結果に直結するため,解析対象の選別や単位整合の監査を行う.

\section{制御設計の妥当性と限界}
ルールベース方策は実装が単純であり,安全側のフォールバックを設計しやすい.一方で,環境ごとに最適な運用点は変化するため,固定閾値では追従できない場合がある.また,切替頻度の増加はログ整合や同期の難易度を上げるため,実装面での制約もある.

\subsection{2値制御の位置づけ}
2値制御は,実機検証を最小構成で成立させる上で有効である.一方で,\SI{1000}{\milli\second}や\SI{2000}{\milli\second}など長間隔を含めた最適化は,より大きな省電力の可能性を持つ.今後は,計測・同期の堅牢性を維持した上で行動空間を拡張し,QoS制約の境界をより広い範囲で評価する必要がある.

\subsection{閾値設計と感度}
CCS閾値は,入力データ(活動種別やモデルの確信度分布)に依存する.閾値を厳しくしすぎると短間隔に偏って省電力効果が薄れ,緩くしすぎると長間隔に偏ってQoS違反が増える.したがって,オフライン評価で候補を絞り込んだ上で,実機の端末内比較で境界付近を詰めるアプローチが実務上有効である.

\section{sleepの扱いと計測公平性}
本研究は,広告間隔を制御することで無線イベント回数を減らし,さらに送信側がsleepに入れる時間を増やすことで平均電力を下げることを狙う.一方で,sleepは計測条件そのものでもあるため,条件が混ざると比較が成立しない.本節では,sleepを議論に入れるための前提を整理する.

\subsection{送信側sleep(平均電力への寄与)}
送信側(DUT)がlight sleep等に入る構成では,広告間隔を延ばすほどsleep時間が増え,平均電力が低下しやすい.ただし,周辺回路やロギングの定常負荷が支配的な場合,改善は飽和する.したがって,sleepの有無(ON/OFFを含む)を条件として明示し,同一コード系列・同一設定の端末内比較として評価することが必要である.

\subsection{受信側sleep(scan dutyの縮退)}
受信側(スマートフォン)のスキャンは,OSの省電力機構や端末状態により間欠化し得る.この受信側sleepは,TL分布の裾を伸ばし,$P_{\mathrm{out}}(\tau)$を悪化させる方向に作用する.したがって,本研究の主張は「理想スキャン下の最適性」ではなく,「非理想スキャン下でも端末内比較として省電力な運用点が存在し得る」ことに置く.

\subsection{OFF計測と差分評価}
計測系の定常負荷が大きい場合,平均電力だけでは広告間隔の差が埋もれる.この場合,広告ON/OFF差分$\Delta E$を広告回数で正規化した$\Delta E/N_{\mathrm{adv}}$を併用することで,「無線1回当たりの増分」として議論できる.ただし,OFFのsleep状態やスタック状態がONと異なると差分の解釈が難しくなるため,Runbookに従って条件を固定する必要がある.

\section{Safe Contextual Banditへの拡張}
将来的には,コンテキスト(不確実度,安定度)に応じて行動(広告間隔)を選択し,報酬(省電力)を最大化しながら制約($P_{\mathrm{out}}(\tau)\le\delta$)を満たすSafe Contextual Banditとして定式化する.ルールベース方策と固定間隔の基準データは,Warm-Startの事前情報として活用できる.

\subsection{Warm-Startの意義}
オンライン学習では,学習初期の探索によってQoS制約を破るリスクがある.ルールベース方策は,安全側の初期方策として利用でき,探索の範囲を制約内に保つ設計に接続しやすい.したがって,Phase 1に相当するルールベース評価は,オンライン最適化の前提条件として重要である.

\section{失敗・未完タスクの位置づけ}
修士論文では,うまくいかなかった点や未完タスクを隠すのではなく,「何が揃っていて,何が揃っていないか」を明示することが再現性に直結する.本研究は,Phase 1として評価系を確立することを目的に置くため,未完点を将来課題として整理し,Phase 2へ接続する材料とする.

\subsection{HARモデル(A\_tiny)の未完点}
A\_tinyは,4クラス性能の未達やTFLite生成・実機計測が未完である.この状態で閾値制御を議論すると,確率出力の較正や量子化誤差により,CCS分布と閾値の意味が変化する危険がある.したがって,本論文では参照モデルA0を土台として不確実度定義を固定し,A\_tinyの改善は今後の課題として明確化する.

\subsection{計測系トラブルの位置づけ}
計測系のトラブル(SYNC取りこぼし,UART欠損,SD失敗など)は,研究の本質ではないが,無視すると結論を逆転させ得る.そのため,本研究では計測系の破綻モードと対策を\secref{sec:measurement_system}で整理し,評価の前提として本文に明示した.

\section{脅威と限界(Threats to Validity)}
本研究の評価は,実機の端末内比較と,オフライン合成評価を組み合わせている.したがって,以下の脅威と限界を明示する必要がある.
\begin{itemize}
  \item 外的妥当性:受信端末やOSバージョンが変わると,スキャンの非理想性が変化し,同一方策でもQoSが変化し得る.
  \item 内的妥当性:ログ欠損や単位換算の誤りは結論を逆転させ得るため,計測健全化と監査を必須とする(\secref{sec:measurement_system}).
  \item モデル近似:オフライン評価は滞在比率に基づく合成であり,切替直後の非定常性や相関は近似に含まれない.
  \item 不確実度較正:確率値が過信されると閾値の意味が崩れるため,較正(温度スケーリング等)の導入が今後の課題となる(\secref{sec:har_uncertainty}).
\end{itemize}

\section{今後の課題}
本研究で未解決の課題として,スキャン挙動の端末差の系統的整理,動的制御におけるQoS指標(TL分布・$P_{\mathrm{out}}$)の高精度評価,およびオンライン学習導入時の安全側設計が挙げられる.

\subsection{QoS評価の高精度化}
動的制御では,イベント時刻の定義と時間同期が結果を左右する.特に長間隔では,開始位相のずれや受信の間欠性により,TL分布と$P_{\mathrm{out}}(\tau)$が不自然に見える場合がある.今後は,ペイロードにインデックス等を埋め込み,時間軸を確実に復元できるログ設計と解析を統一する.

\subsection{行動集合の拡張}
2値制御は最小構成として有効であるが,運用点の探索範囲を狭める.今後は,$\{100,500,1000,2000\}$のように行動集合を拡張し,制約$\delta$の境界に沿ってより良いトレードオフ点が存在するかを検証する.その際,切替頻度が増えるため,ログ整合と時間同期の堅牢化が前提条件となる.
