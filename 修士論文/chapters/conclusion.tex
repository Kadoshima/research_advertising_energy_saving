% chapters/conclusion.tex --- おわりに(番号なし、目次には載せる)
\chapterwithtoc{おわりに}

本研究では,HARの推論不確実度に基づくBLE広告間隔の適応制御を対象とし,ルールベース方策の定義と評価指標の確立,および動的切替の実機検証に取り組んだ.特に,送信・受信・電力計測を統合した計測基盤を整備し,$\si{\micro\coulomb}/\text{event}$と受信遅延に基づく期限超過率を同時に評価できる形に整理した.この結果,単純な閾値ルールでも固定間隔より省電力になりうる運用点が存在し得ることを示し,将来のSafe Contextual Bandit最適化に接続する基盤を与えた.

\section*{本研究のまとめ}
本研究で得られた知見を以下にまとめる.
\begin{itemize}
  \item 不確実度と安定度から複合スコアCCSを構成し,閾値・ヒステリシス・最小滞在時間を用いたルールベース方策として定義した.
  \item 送信(TX),電力(TXSD),受信(RX)の三ノード構成により,同一試行で電力とQoS(TL,$P_{\mathrm{out}}(\tau)$)を評価できる基盤を整備した.
  \item TL/Poutの算出において,開始位相ずれが結果を歪め得ることを示し,定数オフセット補正(v5)として指標定義を固定した.
  \item オフライン評価により,固定間隔点と候補方策点のトレードオフを可視化し,実機実験の探索空間を縮小する枠組みを示した.
  \item 実機実験(D2b/D3)とアブレーション(D4/D4B)により,方策が固定間隔の間の運用点として成立し得ることを示し,さらに$U$が電力(短間隔滞在量)を,CCSが同電力でQoSを改善する役割分担を示唆した.
\end{itemize}

\section*{今後の課題}
今後は,動的制御に対してTL分布と$P_{\mathrm{out}}(\tau)$を高精度に算出し,オフライン評価の予測と実測の差分を体系的に説明できる形にまとめる.また,行動集合を拡張し,制約$\epsilon$の境界に沿ってより良い運用点を探索する必要がある.さらに,Safe Contextual Banditによるオンライン最適化へ拡張し,環境変化に対してもQoS制約を維持しながら省電力化できる枠組みを検証する.
