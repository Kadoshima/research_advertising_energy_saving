% chapters/ch0_introduction.tex --- 第1章 序論
\chapter{序論}
\clearpage

\section{研究背景と社会的文脈}
ウェアラブル端末やエッジデバイスにおいて,推論結果を近傍のスマートフォンへ低遅延に通知する仕組みは,見守りや労働安全など多くの応用で重要である.一方で,常時高頻度に通信を行うと電力消費が増大し,小型バッテリでの長時間運用が困難となる.Bluetooth Low Energy(BLE)の広告(Advertising)は接続を前提としない軽量な近傍通信手段として広く利用されているが,受信側(スマートフォン)のスキャン動作はOSや端末状態に依存し,連続スキャンが保証されない.

\section{課題設定と狙い}
本研究は,アプリケーション層(HAR)で得られる推論不確実度を通信層の制御に利用し,QoS制約(一定時間以内に受信されること)を満たしながら省電力化する枠組みを扱う.推論不確実度と時間的安定度から複合スコアCCSを構成し,その値に応じて広告間隔を段階的に切り替えるルールベース方策を定義する.さらに,送信・受信・電力計測を統合した計測基盤を整備し,電力指標とQoS指標を同一試行で評価できる形に整理する.

\section{問題意識:エッジHARの「推論」と「通信」のねじれ}
ウェアラブル端末やエッジデバイスは,慣性センサ等から人の状態を推定し,異常や重要な変化を近傍のスマートフォンへ通知する用途で用いられる.このとき,アプリケーション層では「推論ができた瞬間に,できるだけ早く届けたい」という要求がある一方,通信層では「常に高頻度に送ると電力が増える」というトレードオフがある.

さらに,BLE広告は送信側の制御が容易である反面,受信側(スマートフォン)のスキャン挙動はOSや端末状態に依存し,連続スキャンが保証されない.したがって,広告間隔を固定して設計した場合,環境変化により「必要なときに届かない」か「届くが電力を使いすぎる」状況が生じ得る.

本研究は,このねじれを緩和するために,アプリケーション層(HAR)で得られる推論不確実度を通信制御に利用し,「いま確信が高いか低いか」に応じて送信頻度を調整する枠組みを扱う.

\section{本研究の核となる主張}
本研究の核となる主張は,次のとおりである.
\begin{quote}
HAR不確実度に基づくBLE広告間隔の適応制御において,単純な閾値ルールでも固定間隔より省電力になりうることを示し,将来のSafe Contextual Bandit最適化の基盤を与える.
\end{quote}

\section{研究課題:QoS制約下での省電力化}
本研究の目的は,受信遅延に基づくQoS制約を満たしつつ,電力消費を抑制することである.代表的なQoSとして,イベント発生後に最初に受信されるまでの遅延(TL)を用い,期限$\tau$を超える確率を
\begin{equation}
  P_{\mathrm{out}}(\tau)=P(\mathrm{TL}>\tau)
\end{equation}
として定義する(詳細は\secref{sec:metrics_detail}).このとき,広告間隔$a$の選択(固定または動的切替)を含む方策$\pi$に対して,以下の形で整理できる.
\begin{equation}
  \text{minimize}\quad \mathbb{E}[q_{\mathrm{event}}(\pi)]
  \quad \text{subject to}\quad P_{\mathrm{out}}(\tau;\pi)\le \epsilon
\end{equation}
ここで$q_{\mathrm{event}}$(単位:$\si{\micro\coulomb}/\text{event}$)は,セッション中の電荷消費をイベント数で正規化した指標である.平均電力$\overline{P}$や$\Delta E/N_{\mathrm{adv}}$等は従属指標として併用し,単位監査や原因切り分けに用いる(\secref{sec:metrics_detail}).

\section{本研究の貢献}
本研究の貢献を以下にまとめる.
\begin{enumerate}
  \item 推論不確実度$U$と安定度$S$から複合スコアCCSを構成し,広告間隔を段階的に切り替えるルールベース方策として定義した.
  \item 送信(TX)・電力(TXSD)・受信(RX)の三ノード構成と同期信号(SYNC/TICK)により,同一試行で電力とQoS(TL,$P_{\mathrm{out}}(\tau)$)を評価できる計測基盤を整備した.
  \item TL/Poutの算出において開始位相ずれが結果を歪め得ることを示し,定数オフセット補正(v5)として指標定義を固定した.
  \item オフライン評価(mHealth合成)により,固定間隔点と候補方策点のトレードオフ(制約帯・Pareto)を可視化し,実機実験の探索空間を縮小する枠組みを示した.
  \item Phase 2の拡張として,$U,S$をコンテキスト,広告間隔を行動,$P_{\mathrm{out}}(\tau)$を制約とするSafe Contextual Banditへ接続できる形で問題設定を整理した.
\end{enumerate}

\section{スコープと前提(Phase 1 / Phase 2)}
本研究は,Phase 2でのSafe Contextual Bandit最適化を見据えつつ,Phase 1として「決め打ちルールで評価指標と計測系を確立する」段階に焦点を当てる.したがって,本論文のスコープは次のとおりである.

\subsection{IN(本論文で扱うこと)}
\begin{itemize}
  \item $U,S,\mathrm{CCS}$の定義とログ化(推論不確実度を通信制御へ接続するための最小要素).
  \item ルールベース方策(閾値,ヒステリシス,最小滞在時間)と,実機での動作確認.
  \item 一次KPI($\si{\micro\coulomb}/\text{event}$,$P_{\mathrm{out}}(\tau)$,TL\_p95)の測定方法と再現性の確立.
\end{itemize}

\subsection{OUT(本論文では深掘りしないこと)}
\begin{itemize}
  \item Safe Banditの学習アルゴリズムそのものの実装・オンライン学習(将来課題として定式化までを示す).
  \item スキャン窓の推定など,受信側OS挙動の同定(非理想スキャン前提で端末内比較を行う).
  \item HARモデルの最終性能の到達(TinyMLの未完タスクは明示し,評価系の枠組みを主張する).
\end{itemize}

\section{本論文の構成}
本論文は,章数を抑えつつ各章を厚くし,特に再現性確保(ログ設計・指標定義・運用手順・破綻モード)を本文内で明示する構成を採用する.第2章では背景と関連研究を統合し,BLE広告・非理想スキャン・sleepの位置づけと関連アプローチを整理する.第3章ではシステム設計と提案手法をまとめ,World Model,一次KPI,HAR不確実度,CCS写像とパラメータを整理する.第4章では計測系・ログ設計・指標定義・Runbook・トラブルシューティングを統合し,再現性のための前提を固定する.第5章ではストレス固定,オフライン評価,実機評価結果をまとめ,固定点と提案方策の比較を示す.第6章では考察として,非理想スキャン下での解釈,sleepの扱い,失敗・未完タスクの位置づけ,およびPhase 2(Safe Contextual Bandit)への展望を述べる.最後におわりにで全体をまとめる.
