% chapters/ch7_offline_eval.tex --- 第7章 オフライン評価(mHealth合成 + ルールベース方策)
\section{オフライン評価:mHealth合成とルールベース方策探索}
\label{sec:offline_eval}

\subsection{目的}
実機実験の探索空間を縮小し,説明可能な代表方策を選定するために,オフライン評価を行う.本節では,mHealth由来の時系列(不確実度・安定度・複合スコア)に対してルールベース方策を適用し,固定間隔の電力テーブルと組み合わせて,QoS制約下の省電力運用点を探索する.

\subsection{入力データ(mHealth合成)}
オフライン評価では,実機上でのHAR推論を直接用いる代わりに,既存データセット由来の時系列(mHealth)から$U,S,\mathrm{CCS}$を合成し,方策を適用する\scite{uci_ml_repo}.これにより,実機で高価な全探索を行う前に,閾値やヒステリシスの候補を絞り込める.

\subsection{方策のパラメータ化}
ルールベース方策は,次のパラメータで表せる.
\begin{itemize}
  \item 閾値:$\theta_{\mathrm{low}},\theta_{\mathrm{high}}$
  \item ヒステリシス幅:$\Delta\theta$(上り・下りの閾値差)
  \item 最小滞在時間:$t_{\mathrm{dwell}}$
  \item 行動集合:$A=\{100,500\}$ または $\{100,500,1000,2000\}$
\end{itemize}
オフライン評価では,これらの組合せをスイープし,平均電力とQoS制約の観点からParetoフロントを抽出する.

\subsection{電力テーブル(固定間隔の実測)}
固定間隔ごとの平均電力は,実機計測から得られる.オフライン評価では,この電力テーブルに基づき,「ある方策が各広告間隔にどの程度滞在するか」を合成して平均電力を推定する.

\subsubsection{平均電力の合成}
方策$\pi$が間隔$a\in A$に滞在する比率を$\rho_\pi(a)$とし,固定間隔の平均電力を$P(a)$とすると,方策の平均電力は
\begin{equation}
  \overline{P}_\pi=\sum_{a\in A}\rho_\pi(a)\,P(a)
\end{equation}
で近似できる.ここで$\rho_\pi(a)$は,mHealth時系列に方策を適用した結果の滞在時間比率として求める.

\subsection{QoS制約と境界条件}
オフライン評価では,代表的な期限$\tau$に対する$P_{\mathrm{out}}(\tau)$を制約として扱い,許容値$\epsilon$を満たす運用点を可行集合として抽出する.$\epsilon$は厳しすぎると可行集合が空になり,緩すぎると省電力とQoSの議論が弱くなるため,境界付近を主張点として設計することが重要である(例:$\epsilon=0.1$).

\subsubsection{QoSの推定}
オフライン評価におけるQoS推定は,固定間隔の実測から得られる$P_{\mathrm{out}}(\tau\mid a)$等の指標を参照し,方策の滞在比率に基づき合成する(詳細は\secref{sec:metrics_detail}).動的切替時の非定常性(切替直後の遅延分布など)は近似に含まれないため,実機評価で差分を検証する必要がある.

\subsection{Paretoフロントと制約帯プロット}
図\ref{fig:letter_delta_band}に,オフライン評価の代表例として,Pout(1s)と平均電力のトレードオフを示す.固定間隔点と候補方策点を同一平面で可視化することで,制約$\epsilon$付近における省電力運用点の存在を示す.

\begin{figure}[tb]
  \centering
  \includegraphics[width=0.95\linewidth]{../results/mhealth_policy_eval/letter_v4_scan90_v5_delta_tight_sleep_on_n9_10_actions_100_500/fig_delta_band.png}
  \caption{制約帯(例:$\epsilon=0.1$)におけるPout(1s)と平均電力のトレードオフ(オフライン評価)}
  \label{fig:letter_delta_band}
\end{figure}

\subsubsection{Paretoプロット(全体像)}
図\ref{fig:pareto_plots}に,スイープ結果の全体像としてParetoプロットの例を示す.ここでは,行動集合を$\{100,500\}$に縮退した評価(実装の単純性を優先)を例とする.

\begin{figure}[tb]
  \centering
  \includegraphics[width=0.95\linewidth]{../results/mhealth_policy_eval/pareto_front_v8_power_table_scan90_v5_sleep_on_n9_10_actions_100_500/pareto_plots.png}
  \caption{Paretoプロット(例):行動集合$\{100,500\}$に縮退したオフライン評価}
  \label{fig:pareto_plots}
\end{figure}

\subsection{行動空間の縮退(\{100,500\})}
実機の動的制御を最小構成で成立させるため,行動空間を\{100,500\}に限定して代表方策を再選定する.この縮退により,実機側の実装・解析の複雑性を抑えつつ,電力低下の主効果(短間隔滞在の削減)に整合した比較が可能となる.

\subsection{選定方策の例}
オフライン評価の出力は,単一の最良方策ではなく,「制約$\epsilon$付近で成立する候補集合」として整理する.その上で,実機実験の本数制約や説明容易性を踏まえて代表点を選定する.代表点の一覧は\secref{sec:extra_results}に示す.

\subsection{まとめ}
本節では,mHealth合成と固定間隔の実測電力テーブルを組み合わせたオフライン評価により,QoS制約下での省電力運用点を探索する枠組みを示した.次節では,実機評価結果(\secref{sec:evaluation_results})を示す.
