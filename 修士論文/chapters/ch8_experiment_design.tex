% chapters/ch8_experiment_design.tex --- 第9章 実験設計
\section{実験設計}
\label{sec:experiment_design}

\subsection{実験の目的}
Phase 1の実験目的は,ルールベース方策(CCS→広告間隔切替)が,固定間隔に対して省電力とQoSのトレードオフ上で優位な運用点になり得ることを示すことである.具体的には,以下を検証する.
\begin{enumerate}
  \item 省電力:固定\SI{100}{\milli\second}よりも$q_{\mathrm{event}}$($\si{\micro\coulomb}/\text{event}$)が低い運用点が存在する.
  \item QoS維持:代表的な期限$\tau$に対して$P_{\mathrm{out}}(\tau)$が過度に悪化しない.
  \item トレードオフ:固定点と動的点を同一平面($P_{\mathrm{out}}(\tau)$--$q_{\mathrm{event}}$)で比較し,Pareto的な位置づけを示す.
\end{enumerate}

\subsection{実験条件}
\subsubsection{環境条件}
実験は,距離,受信端末,スキャン設定等を固定し,端末内比較として実施する.干渉状況の違い(例:E1/E2)は外的要因として扱い,必要に応じて環境を分けて評価する.

\begin{table}[tb]
  \centering
  \caption{環境条件(例)}
  \label{tab:env_conditions}
  \begin{tabular}{ll}
    \toprule
    項目 & 値(例) \\
    \midrule
    距離 & \SI{1}{\meter} \\
    TxPower & \SI{0}{\decibelmilliwatt} \\
    受信端末 & Android(同一端末で統一) \\
    スキャンモード & LOW\_LATENCY(前提条件を固定) \\
    \bottomrule
  \end{tabular}
\end{table}

\subsubsection{制御条件}
制御条件は,固定間隔(ベースライン)と動的制御(提案)を含む.Phase 1の基本方針は,行動集合を$\{100,500,1000,2000\}\,\si{\milli\second}$とし,短間隔(QoS重視)と長間隔(省電力重視)の間を段階的に探索できる形にすることである.

一方で,実機実験の初期段階では,計測系の同期とログ整合を優先し,行動集合を$\{100,500\}\,\si{\milli\second}$に縮退した2値切替で動作成立を確認する.以降,計測の堅牢性が担保できた段階で,$\{100,500,1000,2000\}\,\si{\milli\second}$へ拡張する.

\begin{table}[tb]
  \centering
  \caption{制御条件(例)}
  \label{tab:control_conditions}
  \begin{tabular}{lll}
    \toprule
    条件ID & 条件名 & 広告間隔 \\
    \midrule
    C1 & FIXED-100 & \SI{100}{\milli\second} \\
    C2 & FIXED-500 & \SI{500}{\milli\second} \\
    C3 & FIXED-2000 & \SI{2000}{\milli\second} \\
    C4 & POLICY(2値) & \SI{100}{\milli\second}/\SI{500}{\milli\second} \\
    C5 & POLICY(3値) & \SI{100}{\milli\second}/\SI{500}{\milli\second}/\SI{2000}{\milli\second} \\
    \bottomrule
  \end{tabular}
\end{table}

\subsubsection{反復設計}
統計的な揺らぎと端末状態のばらつきを抑えるため,各条件は複数回反復する.最小の成立確認としては各条件$n\ge3$とし,差が小さい場合に反復数を増やす.

また,運用シナリオに近い評価としては,環境(例:E1/E2)を分け,3条件(例:FIXED-100/FIXED-2000/POLICY)$\times$ 2環境 $\times$ 10セッション=60セッション程度を計画し,端末内比較の前提を維持しながら分散(端末状態の非定常)を平均化する.

\subsection{セッション構成とイベント定義}
\subsubsection{セッション}
セッションは一定時間の試行(trial)の集合として構成する.実機の動的切替では,短い試行(例:\SI{60}{\second})を連続して取り,条件間比較を端末内で完結させる設計が有効である.一方,運用シナリオに近い評価として,\SI{10}{\minute}程度の長いセッションを用い,活動遷移イベントでTL/Poutを評価する設計も重要である.

\subsubsection{イベント}
イベントは,状態遷移(活動の切替)など「QoS評価の基準点」となる時刻である.固定間隔では広告イベントの列から遅延分布を評価できるが,動的切替では開始位相ずれや切替タイミングが影響するため,truthに基づくイベント定義と時間同期が必要となる(次章).

\subsection{指標算出}
\subsubsection{電力指標}
一次KPIの省電力指標は$q_{\mathrm{event}}$($\si{\micro\coulomb}/\text{event}$)である.$q_{\mathrm{event}}$は,TXSDログの電流系列を積分して得た総電荷を,イベント数$N_{\mathrm{event}}$で正規化して算出する(定義は\secref{sec:metrics_detail}).必要に応じて広告OFFの基準を差し引き,広告による増分を評価する.

平均電力$\overline{P}$や総エネルギー$E$,および広告回数で正規化した$\Delta E/N_{\mathrm{adv}}$等は従属指標として併用し,計測健全性の監査(単位・欠損)と原因切り分けに用いる.

\subsubsection{TLと$P_{\mathrm{out}}(\tau)$}
イベント$j$の遅延を$\mathrm{TL}_j$とすると,
\begin{equation}
  P_{\mathrm{out}}(\tau)=\frac{1}{N_{\mathrm{event}}}\sum_{j=1}^{N_{\mathrm{event}}}\mathbb{I}[\mathrm{TL}_j>\tau]
\end{equation}
である.実装では,RXログの時刻とtruthの時刻の定数オフセットを補正し,$\mathrm{TL}_j$を計算する(次章).

\subsubsection{理論モデル(参考)}
固定間隔における$P_{\mathrm{out}}(\tau)$は,1広告の受信成功確率$p_d$を仮定すると,
\begin{equation}
  P_{\mathrm{out}}(\tau\mid a)\approx(1-p_d)^{\lfloor \tau/a \rfloor}
\end{equation}
のように近似できる.ただし,非理想スキャンや開始位相の影響を含むため,本研究では理論値は補助的に用い,主張は実測に基づく.

\subsection{リスクと対策}
実験の主要リスクと対策を以下にまとめる.
\begin{itemize}
  \item 同期ずれ:SYNC/TICKの取りこぼしにより試行区間が崩れる.対策として,preambleで条件IDを通知し,ログ末尾のsummaryで監査する.
  \item 条件混在:LED/SYNCの扱い等が異なるコードを跨いで比較すると定常オフセットが入る.比較は同一コード系列で統一する.
  \item 欠損・単位不整合:UART/SDボトルネックや単位換算ミスは結論を逆転させ得る.パススルー化と再積分監査を行う.
  \item 端末状態の変動:受信端末のOS状態は非定常である.端末内比較を徹底し,実験前チェックリスト(\secref{sec:runbook})を運用する.
\end{itemize}

\subsection{まとめ}
本節では,Phase 1における実験目的と条件,セッション構成,指標算出,およびリスク対策を整理した.次章では,TL/Poutの定義を実測に基づいて固定し,時間同期の手順(v5)を示す.
