% chapters/ch2_proposed.tex --- 第6章 提案手法(CCS→広告間隔制御)
\section{提案手法:CCSに基づくBLE広告間隔制御}
\label{sec:proposed_method}

\subsection{問題設定}
本研究は,広告間隔$a$を選択する方策$\pi$に対して,電荷指標$q_{\mathrm{event}}$を小さくしつつ,受信遅延に基づく期限超過率$P_{\mathrm{out}}(\tau)$を制約する問題として整理する(問題設定は\secref{sec:measurement_setup}と\secref{sec:metrics_detail}に従う).

\subsection{不確実度と安定度}
HARモデルの出力確率分布から不確実度$U\in[0,1]$を導出し,過去一定窓におけるラベル遷移回数から安定度$S\in[0,1]$を導出する(算出方法と前提は\secref{sec:har_uncertainty}).

\subsection{複合スコアに基づく広告間隔制御}
不確実度と安定度から複合スコア(Composite Confidence Score:CCS)を構成し,その値に応じて広告間隔を段階的に切り替える.本研究では,実装の単純性と説明容易性のためにルールベース方策を採用し,閾値とヒステリシス,最小滞在時間により切替の振動を抑制する.

\subsubsection{複合スコアCCS}
CCSは,不確実度$U$と安定度$S$の線形結合として定義し(\secref{sec:har_uncertainty}),閾値写像により広告間隔へ変換する.

\subsubsection{閾値写像}
CCSを3段階に分け,広告間隔$a\in\{100,500,2000\}\,\si{\milli\second}$へ写像する.閾値を$\theta_{\mathrm{low}}$,$\theta_{\mathrm{high}}$($\theta_{\mathrm{low}}<\theta_{\mathrm{high}}$)とすると,
\begin{equation}
  a(\mathrm{CCS}) =
  \begin{cases}
    2000\,\si{\milli\second} & (\mathrm{CCS}\ge\theta_{\mathrm{high}})\\
    500\,\si{\milli\second}  & (\theta_{\mathrm{low}}\le\mathrm{CCS}<\theta_{\mathrm{high}})\\
    100\,\si{\milli\second}  & (\mathrm{CCS}<\theta_{\mathrm{low}})
  \end{cases}
\end{equation}
と定義する.

\subsubsection{ヒステリシスと最小滞在時間}
CCSは推論誤差やセンサノイズにより揺らぐため,単純な閾値写像では切替が過度に発生しうる.そこで,閾値にヒステリシス(上下で異なる閾値)を持たせる.また,最小滞在時間を設け,滞在時間が一定値未満の間は切替を抑制する.

なお,Phase 1で用いる代表的なパラメータ($\alpha,W,\theta_{\mathrm{low/high}}$,ヒステリシス幅,最小滞在時間)は\secref{sec:ccs_params}に一覧化する.

\subsubsection{切替規則(実装上の要点)}
実装は,現在の間隔$a_t$とCCSの観測値$\mathrm{CCS}_t$から次の間隔$a_{t+1}$を決定する有限状態機械として表せる.本研究では以下を満たす設計とする.
\begin{itemize}
  \item \textbf{単調性}:CCSが高いほど長間隔側を選びやすい.
  \item \textbf{安全側への退避}:QoS悪化の兆候がある場合は短間隔へ戻す余地を残す.
  \item \textbf{振動抑制}:ヒステリシスと最小滞在時間で切替頻度を抑える.
\end{itemize}
切替規則の例は,実験装置・計測方式(\secref{sec:measurement_setup})および実験設計(\secref{sec:experiment_design})で示す条件(閾値,滞在時間)と併せて具体化する.

\subsection{ログと説明可能性(Observability)}
動的制御の評価では,「なぜその間隔を選んだか」を後から説明できることが重要である.特に,本研究は非理想スキャンを前提とするため,受信品質の変動をすべて送信側のせいにできない.したがって,方策の意思決定($U,S,\mathrm{CCS}$)と,使用したパラメータ(閾値,ヒステリシス,最小滞在時間等)をログとして残し,観測可能性を担保する.

\subsubsection{切替理由(reason)を毎回残す}
切替のたびに,閾値越え,ヒステリシス判定,最小滞在時間による抑制,フェイルセーフ退避などの理由を\texttt{\detokenize{reason}}として記録する.これにより,結果図表の背後にある「切替の根拠」を監査できる.

\subsubsection{Tx/Rx/Powerのスキーマ}
TXログは方策の内側($U,S,\mathrm{CCS}$と選択した$a$),RXログは受信時刻とseq,TXSDログは電流・電圧系列とsummaryを保持する.ログ列の詳細は\secref{sec:log_schema}に示し,\secref{sec:runbook}に従って運用条件(端末,スキャン設定,sleepの有無)を固定する.

\subsection{動的切替の最小構成(2値制御)}
実機実験の初期段階では,電力低下の主効果が短間隔滞在の削減にあることを踏まえ,広告間隔を\SI{100}{\milli\second}と\SI{500}{\milli\second}の2値に限定した動的切替を用いる.この最小構成により,計測系の同期とログ整合を保ちつつ,動的切替が期待通りに動作することを確認する.

\subsubsection{2値制御の目的}
2値制御は,行動空間を小さくすることで,実装・解析の複雑性を抑える目的がある.また,\SI{100}{\milli\second}と\SI{500}{\milli\second}の差は平均電力の主効果になりやすく,省電力化の寄与を説明しやすい.
