% chapters/ch2_proposed.tex --- 第2章 提案技術
\chapter{提案技術}

\section{問題設定}
本研究では,広告間隔 $a$ を選択して送信を行うとき,エネルギー指標を最小化しつつ,受信遅延に基づくQoS制約を満たすことを目的とする.代表的には,初回受信遅延TLに対して,期限$\tau$を超える確率を$P_{\mathrm{out}}(\tau)=P(\mathrm{TL}>\tau)$として定義し,$P_{\mathrm{out}}(\tau)\le \delta$を制約とする.

\subsection{目的関数(平均電力とイベント当たり指標)}
電力評価は,試行(trial)期間$T$における総エネルギー$E$に基づき,平均電力$\overline{P}=E/T$で表す.一方で,計測系の定常負荷(ロギング処理,LED等)が比較的大きい場合,無線イベントの寄与が平均電力に埋もれる.このため,本研究では,広告ONとOFFの差分$\Delta E=E_{\mathrm{on}}-E_{\mathrm{off}}$を広告イベント数$N_{\mathrm{adv}}$で正規化した$\Delta E/N_{\mathrm{adv}}$(または電荷$\mu C$)も併用し,無線の寄与を分離して議論する.

\subsection{制約(期限超過率)}
QoSは「状態遷移(イベント)後に一定時間以内に受信されること」として定義する.$j$番目のイベントの初回受信遅延を$\mathrm{TL}_j$とすると,
\begin{equation}
  P_{\mathrm{out}}(\tau)=\frac{1}{N_{\mathrm{event}}}\sum_{j=1}^{N_{\mathrm{event}}}\mathbb{I}[\mathrm{TL}_j>\tau]
\end{equation}
で表せる.本研究では,$\tau=\SI{1}{\second},\SI{2}{\second},\SI{3}{\second}$等の代表値で評価し,$P_{\mathrm{out}}(\tau)\le \delta$を制約として扱う.

\section{不確実度と安定度}
HARモデルの出力確率分布から不確実度$U\in[0,1]$を導出し,過去一定窓におけるラベル遷移回数から安定度$S\in[0,1]$を導出する.具体的な算出方法(窓長,正規化,閾値など)は,第3章で評価系と整合する形で述べる.

\subsection{不確実度U}
HARのクラス事後確率を$p_k$($k=1,\dots,K$)とすると,不確実度$U$は正規化エントロピーとして定義できる\scite{shannon1948}.
\begin{equation}
  U = -\frac{\sum_{k=1}^{K} p_k \log p_k}{\log K}\,,
\end{equation}
ここで$U=0$は確信度が高い状態,$U=1$は不確実な状態に対応する.

\subsection{安定度S}
安定度$S$は,直近の時間窓におけるラベル遷移回数に基づき定義する.窓長を$W$,遷移回数を$n_{\mathrm{trans}}$とすると,簡単な定義の一例は以下である.
\begin{equation}
  S = 1-\min\!\left(1,\frac{n_{\mathrm{trans}}}{W}\right)\,.
\end{equation}
安定期では$S$が1に近く,遷移期では$S$が小さくなる.

\section{複合スコアに基づく広告間隔制御}
不確実度と安定度から複合スコア(Composite Confidence Score:CCS)を構成し,その値に応じて広告間隔を段階的に切り替える.本研究では,実装の単純性と説明容易性のためにルールベース方策を採用し,閾値とヒステリシス,最小滞在時間により切替の振動を抑制する.

\subsection{複合スコアCCS}
本研究では,不確実度$U$と安定度$S$から,以下の線形結合で複合スコアCCSを構成する.
\begin{equation}
  \mathrm{CCS} = \alpha\,(1-U) + (1-\alpha)\,S\,.
\end{equation}
ここで$\alpha\in[0,1]$は重みである.$\alpha$を大きくすると確信度($1-U$)を重視し,小さくすると安定度$S$を重視する.

\subsection{閾値写像}
CCSを3段階に分け,広告間隔$a\in\{100,500,2000\}\,\si{\milli\second}$へ写像する.閾値を$\theta_{\mathrm{low}}$,$\theta_{\mathrm{high}}$($\theta_{\mathrm{low}}<\theta_{\mathrm{high}}$)とすると,
\begin{equation}
  a(\mathrm{CCS}) =
  \begin{cases}
    2000\,\si{\milli\second} & (\mathrm{CCS}\ge\theta_{\mathrm{high}})\\
    500\,\si{\milli\second}  & (\theta_{\mathrm{low}}\le\mathrm{CCS}<\theta_{\mathrm{high}})\\
    100\,\si{\milli\second}  & (\mathrm{CCS}<\theta_{\mathrm{low}})
  \end{cases}
\end{equation}
と定義する.

\subsection{ヒステリシスと最小滞在時間}
CCSは推論誤差やセンサノイズにより揺らぐため,単純な閾値写像では切替が過度に発生しうる.そこで,閾値にヒステリシス(上下で異なる閾値)を持たせる.また,最小滞在時間を設け,滞在時間が一定値未満の間は切替を抑制する.

\subsection{切替規則(実装上の要点)}
実装は,現在の間隔$a_t$とCCSの観測値$\mathrm{CCS}_t$から次の間隔$a_{t+1}$を決定する有限状態機械として表せる.本研究では以下を満たす設計とする.
\begin{itemize}
  \item \textbf{単調性}:CCSが高いほど長間隔側を選びやすい.
  \item \textbf{安全側への退避}:QoS悪化の兆候がある場合は短間隔へ戻す余地を残す.
  \item \textbf{振動抑制}:ヒステリシスと最小滞在時間で切替頻度を抑える.
\end{itemize}
切替規則の例は,第3章の実験条件(閾値,滞在時間)と併せて具体化する.

\section{動的切替の最小構成(2値制御)}
実機実験の初期段階では,電力低下の主効果が短間隔滞在の削減にあることを踏まえ,広告間隔を\SI{100}{\milli\second}と\SI{500}{\milli\second}の2値に限定した動的切替を用いる.この最小構成により,計測系の同期とログ整合を保ちつつ,動的切替が期待通りに動作することを確認する.

\subsection{2値制御の目的}
2値制御は,行動空間を小さくすることで,実装・解析の複雑性を抑える目的がある.また,\SI{100}{\milli\second}と\SI{500}{\milli\second}の差は平均電力の主効果になりやすく,省電力化の寄与を説明しやすい.

\section{評価システム概要(TX/TXSD/RX)}
評価系は,送信(TX),電力ロガ(TXSD),受信(RX)の三ノードで構成する.TXは広告間隔を制御しつつ,必要なメタ情報をペイロードに埋め込む.TXSDはINA219により電流・電圧を計測して平均電力等を算出する.RXは受信ログを蓄積し,PDR,TL,$P_{\mathrm{out}}(\tau)$を評価する.

\begin{figure}[tb]
  \centering
  \fbox{\begin{minipage}{0.9\linewidth}
    \vspace{1mm}
    \begin{center}
      \textbf{TX(DUT)}\hspace{6mm}$\rightarrow$\hspace{6mm}\textbf{RX(スマートフォン)}
    \end{center}
    \vspace{1mm}
    \begin{itemize}
      \item TX:HARの出力から$U,S,\mathrm{CCS}$を計算し,広告間隔$a$を切替.ペイロードにseqや状態を埋め込む.
      \item RX:受信時刻・RSSI・seqを記録し,PDR/TL/$P_{\mathrm{out}}(\tau)$を算出.
      \item TXSD:TXの電流・電圧を計測し,$E$と$\overline{P}$を算出(必要に応じてTICKで$N_{\mathrm{adv}}$をカウント).
    \end{itemize}
    \vspace{1mm}
  \end{minipage}}
  \caption{評価システム概要(TX/TXSD/RX)}
  \label{fig:system_overview}
\end{figure}

本文中では \figref{fig:system_overview} のように参照する.
