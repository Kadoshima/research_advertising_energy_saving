% chapters/appx_h_ccs_params.tex --- 付録H CCS写像とパラメータ(版管理)
\section{CCS写像とパラメータ(版管理)}
\label{sec:ccs_params}

本節では,Phase 1(ルールベース)で用いるCCS写像と主要パラメータを一覧化し,モデル更新時の再キャリブレーション対象を明確化する.評価の再現性のため,パラメータは実験ログ(TX)に保存し,作業ログに変更履歴を残す.

\subsection{3状態写像(ACTIVE/UNCERTAIN/QUIET)}
本研究のルールベース方策は,CCSを3状態へ写像し,広告間隔を段階的に切り替える.表\ref{tab:ccs_state_mapping}に概念整理を示す.

\begin{table}[tb]
  \centering
  \caption{CCS状態と広告間隔の写像(概念)}
  \label{tab:ccs_state_mapping}
  \begin{tabular}{p{0.22\linewidth}p{0.22\linewidth}p{0.46\linewidth}}
    \toprule
    状態 & 広告間隔 $a$ & 意図 \\
    \midrule
    ACTIVE & \SI{100}{\milli\second} & 不確実または遷移期:QoS優先(短間隔) \\
    UNCERTAIN & \SI{500}{\milli\second} & 中間:過渡の揺らぎを吸収しつつ省電力化 \\
    QUIET & \SI{2000}{\milli\second} & 安定期:省電力優先(長間隔) \\
    \bottomrule
  \end{tabular}
\end{table}

\subsection{主要パラメータ(Phase 1既定値)}
表\ref{tab:ccs_params_phase1}に,Phase 1で用いる代表的パラメータを示す.これらはモデルの出力分布や量子化誤差に依存するため,モデル更新時には再キャリブレーションが必要となる.

\begin{table}[tb]
  \centering
  \caption{CCS写像と安定化の主要パラメータ(例)}
  \label{tab:ccs_params_phase1}
  \begin{tabular}{p{0.28\linewidth}p{0.18\linewidth}p{0.44\linewidth}}
    \toprule
    パラメータ & 値(例) & 説明 \\
    \midrule
    $\alpha$ & 0.7 & $\mathrm{CCS}=\alpha(1-U)+(1-\alpha)S$ の重み \\
    $W$ & 5 & 安定度$S$の窓長(遷移回数の正規化) \\
    $\theta_{\mathrm{low}}$ & 0.80 & ACTIVE$\leftrightarrow$UNCERTAINの閾値(上り) \\
    $\theta_{\mathrm{high}}$ & 0.90 & UNCERTAIN$\leftrightarrow$QUIETの閾値(上り) \\
    $h$ & 0.05 & ヒステリシス幅(下りは$\theta-h$) \\
    $t_{\min}$ & \SI{2}{\second} & 最小滞在時間(切替の振動抑制) \\
    行動集合 & $\{100,500,2000\}$ ms & Phase 1の最小構成(拡張候補は$\{100,500,1000,2000\}$) \\
    \bottomrule
  \end{tabular}
\end{table}

\subsection{ログへの記録項目}
方策の説明可能性と再現性のため,少なくとも以下をTXログに記録する.
\begin{itemize}
  \item $U,S,\mathrm{CCS}$と,選択した広告間隔$a$
  \item $\theta_{\mathrm{low/high}}$,ヒステリシス幅$h$,最小滞在時間$t_{\min}$
  \item 切替理由(\texttt{\detokenize{reason}})
\end{itemize}
これにより,結果図表の背後にある意思決定とパラメータを監査できる.
