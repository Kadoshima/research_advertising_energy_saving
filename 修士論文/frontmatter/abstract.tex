% frontmatter/abstract.tex --- 要旨本文
% ここには通常,背景→課題→提案→方法→結果→結論→今後の課題,の順に1ページ程度で書く.

本研究では,Human Activity Recognition(HAR)の推論不確実度に基づいてBluetooth Low Energy(BLE)の広告間隔を適応制御し,省電力化とQuality of Service(QoS)維持を両立する手法を検討する.ウェアラブル端末やエッジデバイスでは,常時通信による電力消費がボトルネックとなる一方,スマートフォン側のスキャンは非理想的であり,広告間隔を単純に長くすると受信遅延や期限超過が増える.

提案手法は,HARの出力確率から得られる不確実度と,時間的安定度から複合スコアCCSを構成し,その値に応じて広告間隔を段階的に切り替えるルールベース方策である.実装・評価のために,送信(TX)・電力ロガ(TXSD)・受信(RX)の三ノード構成を用い,省電力指標$q_{\mathrm{event}}$($\si{\micro\coulomb}/\text{event}$)と,受信遅延に基づく期限超過率$P_{\mathrm{out}}(\tau)$を同一試行で評価できる計測基盤を整備する.

オフライン評価と実機実験により,単純な閾値ルールでも固定間隔より省電力になりうる運用点が存在することを確認する.さらに,\SI{100}{\milli\second}と\SI{500}{\milli\second}の2値切替を用いた最小構成の動的制御を実装し,固定間隔と比較して中間的な挙動を示すことを確認する.最後に,非理想スキャン環境における限界と,Safe Contextual Banditによるオンライン最適化への拡張方針を議論する.
