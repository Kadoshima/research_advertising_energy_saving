% main.tex --- 修論用アブスト(2ページ二段組, LuaLaTeX)
\documentclass[a4paper,10pt,twocolumn]{ltjsarticle}

\usepackage[haranoaji]{luatexja-preset}
\usepackage[top=20mm,bottom=20mm,left=20mm,right=20mm]{geometry}
\usepackage{graphicx}
\usepackage{booktabs}
\usepackage{siunitx}
\usepackage{caption}
\usepackage{titlesec}
\usepackage{balance}
\usepackage[hidelinks]{hyperref}
\usepackage[backend=biber,style=numeric,sorting=none]{biblatex}

% meta.tex --- 論文メタ情報(ここだけ編集すればOK)
% 全角数字にしたい場合は「2025」のように入力してください。

% ===== 表紙に出る情報 =====
\newcommand{\ThesisYearLabel}{2025年度修士論文}
\newcommand{\ThesisTitle}{HAR不確実度に基づくBLE広告間隔の適応制御に関する研究}

\newcommand{\University}{中部大学大学院}
\newcommand{\GraduateSchool}{工学研究科}
\newcommand{\Department}{情報工学専攻}
\newcommand{\Course}{博士前期課程}

\newcommand{\AuthorFamily}{萩原}
\newcommand{\AuthorGiven}{圭島}

\newcommand{\AdvisorFamily}{木村}
\newcommand{\AdvisorGiven}{秀明}

% ===== PDFの「修士論文題目」ページ(要旨ページ)に出る情報 =====
\newcommand{\AbstractDepartment}{情報工学専攻}

\addbibresource{references.bib}

% ページ番号は出さない
\pagestyle{empty}

% 段組間の余白
\setlength{\columnsep}{7mm}

% 段落
\setlength{\parindent}{1\zw}
\setlength{\parskip}{0pt}

% 見出し: "1. はじめに" 形式(太字は避け、サイズで区別)
\titleformat{\section}{\normalfont\normalsize}{\thesection.}{0.6em}{}
\titlespacing*{\section}{0pt}{0.7\baselineskip}{0.4\baselineskip}
\titleformat{\subsection}{\normalfont\normalsize}{\thesubsection}{0.6em}{}
\titlespacing*{\subsection}{0pt}{0.4\baselineskip}{0.2\baselineskip}

% 図表キャプション: "図 1 タイトル" / "表 1 タイトル"(コロン無し)
\captionsetup{labelsep=space}
\captionsetup[figure]{name=図,position=bottom}
\captionsetup[table]{name=表,position=top}

% 参照(図/表)
\newcommand{\figref}[1]{図~\ref{#1}}
\newcommand{\tabref}[1]{表~\ref{#1}}

% [1][2] 形式の引用
\DeclareCiteCommand{\scite}
  {}
  {\mkbibbrackets{\usebibmacro{citeindex}\usebibmacro{cite}}}
  {}
  {}

% URL 表記を簡素化("url:" / "visited on" を出さない)
\DeclareFieldFormat{url}{\url{#1}}
\renewbibmacro*{url+urldate}{\printfield{url}}

% 浮動体の間隔を詰めて2ページに収めやすくする
\setlength{\textfloatsep}{8pt plus 2pt minus 2pt}
\setlength{\intextsep}{6pt plus 2pt minus 2pt}
\setlength{\floatsep}{6pt plus 2pt minus 2pt}

\begin{document}

\twocolumn[
\begin{center}
  {\Large \AbstTitle\par}
  \vspace{1mm}
  {\normalsize \AbstAuthor\quad(指導教員:\AbstAdvisor)\par}
  {\normalsize \AbstAffiliation\par}
  \vspace{3mm}
\end{center}
]

\section{はじめに}
ウェアラブル等のエッジ端末における行動認識(Human Activity Recognition; HAR)では,推論と通信が電力消費を支配しやすい.Bluetooth Low Energy(BLE)広告は簡便なブロードキャスト手段である一方,スマートフォン受信(スキャン)が非理想であるため,固定広告間隔では遅延分布の裾が悪化し得る\scite{bluetooth_core_spec,android_ble_scanner}.本研究は,HARの不確実度に応じてBLE広告間隔を適応させ,期限超過率$P_{\mathrm{out}}(\tau)$を満たしつつ送信側消費を抑えることを目的とする.

\par
固定間隔を短くすると平均消費が増え,長くすると発見遅延が増えるため,単純な固定設定では省電力とQoSの両立が難しい.そこで本研究では,QoS制約を期限超過率$P_{\mathrm{out}}(\tau)$(本稿では$\tau=\SI{1}{\second}$)で表し,これを満たしつつ送信イベント当たり電荷$q_{\mathrm{event}}$(\si{\micro\coulomb}/event)を小さくする設計問題として扱う.

\par
本研究の貢献は,(i) HAR不確実度から広告間隔を切り替えるルールベース制御(最小構成として2値切替)を実装し,(ii) TX/TXSD/RXの三ノード計測により電力とQoSを同一試行で比較できる評価系を整備し,(iii) 実機において固定間隔より良い運用点が存在することを確認した点である.

\section{提案手法}
不確実度の指標にはエントロピー等が知られる\scite{shannon1948}.一方で,実装上は計算量や確率の校正(calibration)の影響が重要であり\scite{guo2017calibration},本研究ではHARモデルの出力確信度と時系列安定度から通信信頼度指標CCS(Communication Confidence Score)を構成する.CCSに基づいて広告間隔を切り替え,本アブストでは実装の最小構成として$\SI{100}{\milli\second}$と$\SI{500}{\milli\second}$の2値切替方策を扱う.CCSが低い(遷移・不確実)区間では短間隔へ寄せてQoSを守り,それ以外は長間隔へ寄せて平均消費を下げる.評価指標は,送信イベント当たり電荷$q_{\mathrm{event}}$(\si{\micro\coulomb}/event)と遅延指標TLおよび期限超過率$P_{\mathrm{out}}(\tau)$とし,平均電力は補助的に用いる.

\subsection{CCSと2値切替}
CCSは,モデル出力の確信度と,推定ラベルの時間的一貫性(安定度)を合成して定義する.具体例として,確信度と安定度の加重和を用い,閾値とヒステリシスにより状態(ACTIVE/QUIET等)を切り替える.これにより,遷移が多い区間ほど短間隔に滞在し,定常区間は長間隔に滞在する設計とする.

\par
切替が閾値のみに依存すると振動し得るため,ヒステリシスに加えて最小滞在時間を設ける.さらに,受信側のQoS悪化が観測された場合は強制的に短間隔へ退避するフェイルセーフを導入し,制約違反を抑制する.

\section{実験}
計測はTX(DUT),TXSD(電力ロガ),RX(受信ロガ)の三ノード構成とし,同期信号で試行区間を揃える.TXSDは電流・電圧をログ化し,試行ごとに総エネルギーと平均電力を算出する.RXは受信時刻とタグを記録し,truth(\SI{100}{\milli\second}格子)と定数オフセットで時間同期した上で,遷移に対するTLと$P_{\mathrm{out}}(\tau)$を算出する.Android端末のscan dutyを90\%とした条件(scan90)で,固定$\SI{100}{\milli\second}$,固定$\SI{500}{\milli\second}$,および方策を比較した(S1/S4,各条件$n=6$).

\par
\figref{fig:measurement_setup}に計測系の外観を示す.

\begin{figure}[t]
  \centering
  \includegraphics[width=\linewidth]{figures/fig_measurement_setup}
  \caption{計測系の外観(TX/TXSD/RXの三ノード構成)}
  \label{fig:measurement_setup}
\end{figure}

\subsection{比較条件}
S1は遷移が少ないシナリオ,S4は遷移が多いシナリオとして扱い,いずれも固定$\SI{100}{\milli\second}$,固定$\SI{500}{\milli\second}$,方策(2値切替)の3条件を比較する.各試行は約\SI{180}{\second}とし,6条件(S1/S4$\times$3)それぞれ$n=6$で統合集計する.

\section{結果}
図\ref{fig:d2b_tradeoff}は,平均電力と$P_{\mathrm{out}}(\SI{1}{\second})$のトレードオフを可視化したものである.主要指標を表\ref{tab:d2b_summary}に示す.方策は固定$\SI{100}{\milli\second}$より低電力であり,固定$\SI{500}{\milli\second}$より低い期限超過率(QoS改善)を示す運用点になり得る.例えば,S1では方策の平均電力は$191.5\,\si{\milli\watt}$で固定$\SI{100}{\milli\second}$($204.1\,\si{\milli\watt}$)より\SI{12.6}{\milli\watt}低く,かつ$P_{\mathrm{out}}(\SI{1}{\second})$は0.125で固定$\SI{500}{\milli\second}$(0.142)より改善した.S4でも同様に,方策は固定$\SI{100}{\milli\second}$より\SI{7.9}{\milli\watt}低電力で,固定$\SI{500}{\milli\second}$より低い$P_{\mathrm{out}}(\SI{1}{\second})$(0.069)を達成した.また,短間隔滞在比率share100はS1で約0.33,S4で約0.59となり,遷移が多い側ほど短間隔へ寄せる挙動が定量で確認できる.

\begin{table}[t]
  \caption{D2b(scan90,$n=6$)の主要指標(平均$\pm$標準偏差)}
  \label{tab:d2b_summary}
  \centering
  \scriptsize
  \setlength{\tabcolsep}{3pt}
  \begin{tabular}{lrrr}
    \toprule
    条件 & avg power [mW] & $P_{\mathrm{out}}(\SI{1}{\second})$ & share100 \\
    \midrule
    S1 fixed100 & 204.1$\pm$1.4 & 0.075$\pm$0.027 & 1.00 \\
    S1 fixed500 & 184.7$\pm$1.5 & 0.142$\pm$0.049 & 0.00 \\
    S1 policy   & 191.5$\pm$1.9 & 0.125$\pm$0.027 & 0.33 \\
    \addlinespace[0.2em]
    S4 fixed100 & 204.4$\pm$2.1 & 0.053$\pm$0.010 & 1.00 \\
    S4 fixed500 & 184.5$\pm$1.5 & 0.146$\pm$0.031 & 0.00 \\
    S4 policy   & 196.6$\pm$1.6 & 0.069$\pm$0.029 & 0.59 \\
    \bottomrule
  \end{tabular}
\end{table}

\begin{figure}[t]
  \centering
  \includegraphics[width=\linewidth]{figures/fig_d2b_power_vs_pout_n6}
  \caption{D2b(scan90,$n=6$)における平均電力と$P_{\mathrm{out}}(\SI{1}{\second})$の関係(share100を注釈)}
  \label{fig:d2b_tradeoff}
\end{figure}

\section{考察}
share100の変化は,遷移(高不確実)時に短間隔へ寄せるという設計意図と整合する.また,平均電力は短間隔滞在比率の線形混合で概ね説明できるため,省電力効果の支配要因を「sleep設定」ではなく「短間隔滞在比率」に帰着できる.一方で,受信側スキャンの非理想性により,PDRだけではQoSを表現しきれないため,今後もTLと$P_{\mathrm{out}}(\tau)$を一次指標として扱う必要がある.

\par
例えば固定条件の平均電力を$P_{100},P_{500}$とすると,方策の平均電力$P$は概ね
\[
  P \approx \mathrm{share100}\,P_{100} + (1-\mathrm{share100})\,P_{500}
\]
で近似できる.この近似が成り立つことは,「いつ短間隔へ戻すか」という制御則設計が電力を規定することを意味し,学習(Bandit)に拡張する際の目的関数設計にも直結する.

\section{まとめ}
HAR不確実度に基づく2値広告間隔制御が,固定間隔の中間的電力でQoSを保つ運用点になり得ることを実機で示した.今後は,広告間隔候補を多段化し,scan dutyが変動する環境でも$P_{\mathrm{out}}(\tau)$制約を満たす方策へ拡張する.さらに,CCSをコンテキストとし,制約下で広告間隔を選択するSafe Contextual Banditへ展開する\scite{lattimore2020bandit}.

\vspace{-1mm}
\balance
\section*{文 献}
\printbibliography[heading=none]

\end{document}
